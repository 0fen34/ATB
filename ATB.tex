% !TeX root = ATB.tex
\documentclass{report}
\usepackage[english, main = ngerman]{babel}
\usepackage{tipa} % IPA Aussprache Symbole
\usepackage{booktabs}   % Verbesserte Möglichkeiten für Tabellenlayout über horizontale Linien
\usepackage{tabulary}
\usepackage{siunitx}
\usepackage{enumerate}
\usepackage[shortlabels]{enumitem}
\usepackage[breaklinks,colorlinks, urlcolor=black, pdfusetitle,linkcolor=black,anchorcolor=black,citecolor=black]{hyperref}
\usepackage[nameinlink]{cleveref}
\usepackage{enumitem}

\counterwithout{section}{chapter}
\counterwithin{enumi}{section}

\title{\Huge Allgemeine Trinkbedingungen \\[2ex] \Large Regelwerk für Trinkspiele und Trinkpflichten}
\author{}
\date{\today}

\addto\captionsngerman{%
	\renewcommand{\abstractname}{Präambel}
}

\begin{document}

\renewcommand{\thesection}{\S\arabic{section}}

% Redefine the format of cross-references for enumeration items
\makeatletter
\renewcommand{\p@enumi}{\thesection{} Absatz }
\makeatother

\pagenumbering{roman}
\maketitle
\begin{abstract}

	Es gibt wohl mehr Hausregeln und eigene Variationen von Trinkspielen als studentische Haushalte in Deutschland.
	Diese Vielfalt ist ein Schatz, an dem man sich bereichern kann und dem gegenüber offen zu bleiben sich stets lohnt.
	
	Treffen jedoch mehrere Haushalte auf einander, so entflammt oft eine leidenschaftliche Debatte über die anzuwendenden Regeln.
	Spätestens im Laufe eines Spiels fallen Unterschiede auf und dann, meist zu spät, wird diskutiert.
	Dieses Dokument soll ebendies verhindern indem ein gemeinsames Grundverständnis, eine gemeinsame Basis, geschaffen wird.

	Spiel- und Verhaltensregeln werden klar definiert um Missverständnisse zu vermeiden.
	Ziel ist es dabei, ein wohlbalanciertes Erlebnis in einer wohlwollenden Umgebung zu erschaffen.
	In abweichenden Kontexten, wie Turnieren, mag es sinnvoll sein, ein anderes Regelwerk zu Rate zu ziehen.
	\\

	Wo sich alle Spielenden einig sind, können Regeln einfach abgewandelt werden.
	Veränderungswünsche an diesem Regelwerk werden gerne in Form von Merge Requests entgegengenommen.
	Bitte habt Verständnis, dass nicht alle Wünsche integriert werden können und Kompromisse gemacht werden müssen.
	\\

	In diesem Sinne: Habt Spaß und passt auf euch und einander auf \begin{math}
		\heartsuit
	\end{math}
	\\
	\\
	\\
	\\
	Weitere Informationen findet ihr hier:
	\url{https://github.com/0fen34/ATB}
\end{abstract}
\tableofcontents
\cleardoublepage{}
\pagenumbering{arabic}

\section{Allgemeine Regelungen}\label{Allgemeine_Regelungen}

\subsection{Ehre}\label{Allgemeine_Regelungen:Ehre}
\paragraph{} Allgemein, aber insbesondere wo keine neutralen Beobachtenden beteiligt sind, verhalten sich die Spielenden als Ehrenmensch.
Es werden keine Regelverletzungen bewusst begangen oder fälschlich angeprangert oder abgestritten.

\paragraph{}\label{Allgemeine_Regelungen:Ehre:Spirit}
Um den \glqq{} Spirit of the Game\grqq{} zu erhalten können Spielende gemeinsam entscheiden Regeln auszusetzen.
Ausgenommen hiervon sind~\ref{Allgemeine_Regelungen:Ehre} und~\ref{Allgemeine_Regelungen:Gender}.

\paragraph{}
~\ref{Allgemeine_Regelungen:Ehre:Spirit} findet insbesondere Verwendung, wenn Personen am Spiel beteiligt sind, die grundsätzlich keinen Alkohol konsumieren möchten.
Diese sollen, wo möglich, den nicht ausgeschlossen werden.

\subsection{Gender}\label{Allgemeine_Regelungen:Gender}
\paragraph{}
Wird in einer Regel zwischen Männern und Frauen differenziert ist das Geschlecht gemeint, dem sich die Spielenden zum Zeitpunkt des Spiels zugehörig fühlen.

\paragraph{}
Menschen, die sich weder Männern noch Frauen zugehörig fühlen, wählen frei, mit welcher Gruppe sie für die Dauer des Spiels gemeinsam trinken.
Diese Wahl ist in keiner Weise zu interpretieren.

\subsection{Das Trinken}
\paragraph{}\label{Allgemeine_Regelungen:Trinken:Trink}
Wer spielt, trinkt.

\paragraph{}
~\ref{Allgemeine_Regelungen:Trinken:Trink} gilt nicht mehr, wenn sich eine Person entscheidet mit dem Trinken aufzuhören.
In diesem Fall kann die Spieler:in ausscheiden oder es wird entweder das Spiel abgebrochen, die Spieler:in ausgetauscht, ein Aussetzen der Trinkpflicht für diese Person entschieden oder die Trinkpflicht von einer freiwilligen Person übernommen.
Die Entscheidung treffen die Spielenden gemeinsam.

\paragraph{}
Spielende können jederzeit aus einem Spiel ausscheiden. Wenn für das Spiel nötig, sollten sie eine Ersatzspieler:in anwerben.

\subsection{Getränke}\label{Allgemeine_Regelungen:Getränke}
\paragraph{}\label{Allgemeine_Regelungen:Getränke:Umrechnung}
Es folgt ein allgemeiner Umrechnungsschlüssel für Getränke.
Wenn Spielende ein Getränk durch ein anderes ersetzen möchten, so können sie dies entsprechend den folgenden Verhältnissen tun.
1l Bier entspricht:

\begin{tabular}{l l}
	1l   & Apfelwein \\
	0,4l & Wein      \\
	0,3l & Pfeffi    \\
	2l   & Radler
\end{tabular}

\paragraph{}\label{Allgemeine_Regelungen:Getränke:Abfüllen}
Werden bereits vor Spielbeginn Trinkeinheiten abgefüllt, so werden diese anteilig nach den Wünschen der Spielenden bereitgestellt.
Ist eine Partei in der Trinkpflicht und hat bereits alle Anteile des eigenen Wunschgetränks aufgebraucht, so beginnt sie die übrigen Anteile nach eigener Wahl zu leeren.

\paragraph{}
Ausnahmen zu~\ref{Allgemeine_Regelungen:Getränke:Abfüllen} wie etwa ersetzten der Trinkeinheiten oder Tauschen mit den Mitspielenden sind möglich.

\paragraph{}
Muss eine Flasche Bier geext werden, steht es der betreffenden Person frei dies in Form eines Tornados zu tun, sofern es die Örtlichkeiten zulassen.
Solange der Versuch ernsthaft durchgeführt und ein Großteil der Flasche geleert wurde, gilt die Trinkpflicht als erfüllt auch, wenn noch Bier in der Flasche ist.

\paragraph{}
Wenn nicht anders spezifiziert beziehen sich Schluckangaben auf Bier.
Spielende mit einem anderen Getränk müssen entsprechend~\ref{Allgemeine_Regelungen:Getränke:Umrechnung} umrechnen.

\section{Bier-Pong}
\subsection{Vorbereitung}
\paragraph{}
Zwei Teams aus je zwei Spielenden spielen gegeneinander.
Auf dem Tisch werden pro Team 10 Becher in Pyramidenform aufgestellt.
Die Basis der Pyramide befindet sich dabei mittig kurz vor den gegenüberliegenden kürzeren Seiten des Tisches und die Becher berühren einander.
Auf jeder Seite wird 1l Bier gleichmäßig auf die Becher verteilt.
Außerdem wird pro Seite ein mit Wasser gefüllter Becher bereitgestellt, um darin die Bälle zu säubern.

\paragraph{}
Variationen der Teamgrößen sind möglich.
Die Füllmengen werden entsprechend angepasst.
Bei ungleichen Teams entscheidet das kleinere Team, ob beide Teams gleich viel Inhalt vorbereiten und wenn ja, an welcher Teamgröße sich orientiert wird.

\paragraph{}
Variationen der Füllmengen sind möglich, wenn sich die Teams darauf einigen.

\paragraph{}
Variationen der Getränke sind entsprechend~\ref{Allgemeine_Regelungen:Getränke} möglich, wenn sich die Teams darauf einigen.

\subsection{Spielablauf}\label{Bier-Pong:Spielablauf}
\paragraph{}
Das Auswerfen: Je eine Spielende aus jedem Team wirft gleichzeitig mit je einem Ball auf die gegnerische Formation.
Sie schauen sich dabei in die Augen. Das Auswerfen wird so lange wiederholt bis genau eines der Teams trifft und somit gewinnt.
In diesem Prozess wechseln sich die Teammitglieder ab.
Das Sieger-Team beginnt das Hauptspiel und hat am Ende einen Nachwurf.
Das Verlierer-Team hat keinen Nachwurf.

\paragraph{}
Das Hauptspiel: Die Teams werfen in einem Zug nacheinander je zwei Bälle auf die gegnerische Formation am gegenüberliegenden Tischende.
Um als Treffer zu zählen, muss der Ball im Becher zur Ruhe kommen.
Die Becher bleiben mitsamt sich womöglich darin befindenden Bällen bis zum Ende eines Zuges unberührt stehen.
Der Zug endet, wenn das werfende Team keine Würfe mehr offen hat.

\paragraph{}
Das Spielende: Wurden alle Becher eines Teams getroffen und hat dieses Team kein Wurfrecht mehr, so hat es das Spiel verloren.
Das Verlierer-Team teilt die übrigen Becher des Siegerteams unter sich auf und trinkt diese zeitnah.

\paragraph{}
\glqq{} Sudden Death\grqq{}: Wurden alle Becher aller Teams getroffen ohne ein klares Siegerteam, wird der Sieg durch Sudden Death entschieden.
Auf jeder Seite wird ein einzelner Becher aufgestellt und zur Hälfte gefüllt.
Die Teams werfen abwechselnd mit je einem Ball.
Wer als erstes trifft ohne, dass das andere Team ebenfalls im Nachwurf trifft, gewinnt das Spiel.

\subsection{Das Werfen}
\paragraph{}\label{Bier-Pong:Werfen:Ellenbogen}
Beim Werfen muss der Ellenbogen des Wurfarms ständig hinter der Tischkante bleiben.

\paragraph{}
Wird~\ref{Bier-Pong:Werfen:Ellenbogen} verletzt, so wird zunächst eine Verwarnung ausgesprochen und der Wurf wiederholt.
Bei wiederholter Verletzung muss außerdem ein zusätzlicher Becher getrunken werden.

\paragraph{}
\glqq{} Trickshots\grqq{} müssen angekündigt werden um als solche zu zählen.
Sie müssen sich signifikant von normalen Würfen unterscheiden.
Trifft ein Trickshot, so muss ein zusätzlicher Becher getrunken werden.

\paragraph{}
\glqq{} Bouncer\grqq{} müssen nicht angekündigt werden.
Prallt ein Ball von der Tischplatte ab, bevor er einen Becher trifft, so müssen ebenso viele zusätzliche Becher getrunken werden, wie der Ball aufgeprallt ist.
Bouncer dürfen entsprechend~\ref{Bier-Pong:Verteidigen} abgewehrt werden.

\paragraph{}
Als \glqq{} Airball\grqq{} zählt jeder Wurf außer Trickshots, der am Tisch vorbei geworfen wird, ohne etwas zu berühren.
Wird der Airball vom verteidigenden Team vor deren nächsten Wurf erkannt, hat die werfende Person die Wahl entweder einen Shot zu trinken oder bis Spielende ein Kleidungsstück auszuziehen.
Socken oder ähnliches zählen gemeinsam als ein Kleidungsstück.
Schmuck zählt nicht zur Kleidung.
Die Wahl steht der werfenden Person frei.

\paragraph{}\label{Bier-Pong:Werfen:Ankündigen}
Es ist möglich einen konkreten Becher als Ziel anzugeben.
Wird der Becher getroffen, muss ein zusätzlicher Becher getrunken werden.
Wird dieser Becher nicht getroffen, trinkt stattdessen die werfende Person einen eigenen Becher ihrer Wahl.

\paragraph{}
Ziele gemäß~\ref{Bier-Pong:Werfen:Ankündigen} können nur angekündigt werden, wenn noch mindestens vier Becher zu treffen sind.

\paragraph{}
Gastwürfe von nicht spielenden Personen sind erlaubt, wenn das gegnerische Team sein Einverständnis gibt.

\paragraph{}
Sammeln sich im Laufe eines Zuges zusätzliche zu trinkende Becher an, so werden die zu trinkenden Becher am Ende des Zuges vom werfenden Team bestimmt und zählen ab diesem Moment als getroffen.

\subsection{Verteidigen}\label{Bier-Pong:Verteidigen}
\paragraph{}
Wird unerlaubt verteidigt, so wird der Wurf wiederholt und es muss ein zusätzlicher Becher getrunken werden.

\paragraph{}
Verschüttete eigene Becher müssen wieder aufgefüllt und anschließend getrunken werden.
Werden beim Verteidigen gegnerische Becher verschüttet, so werden diese wieder aufgefüllt und es muss die entsprechende Anzahl eigener Becher getrunken werden.
Die Auswahl erfolgt hierbei frei.

\paragraph{}
Als Verteidigen gilt jedes signifikante Einwirken auf den Ball.

\paragraph{}
Verteidigen ist grundsätzlich nicht erlaubt.
Fingern und Blasen sind ausdrücklich nicht erlaubt.
Der Tisch oder die Becher dürfen während eines gegnerischen Wurfes ausdrücklich nicht bewegt werden.
Dies würde als unerlaubtes Verteidigen zählen.

\paragraph{}
Bälle, die bereits Kontakt mit einem anderen Objekt hatten dürfen verteidigt werden.
Hierzu zählen insbesondere Bouncer.
Trickshots dürfen nicht verteidigt werden.

\paragraph{}
Airballs dürfen gefangen werden, sobald sie über die Tischkante geflogen sind.
Einen Airball vorher abzufangen zählt als unerlaubte Verteidigung.

\subsection{Getroffene Becher}
\paragraph{}
Getroffene Becher werden direkt nach Zugende aus der Formation entfernt und zeitnah geleert.
Üblicherweise trinken die Teammitglieder abwechselnd.
Sie können sich aber auch auf eine alternative Regelung einigen.

\paragraph{}
Durch den Wurf umgeworfene Becher zählen nicht als getroffen.
Sie werden nach Ende des Zuges wieder neu befüllt.

\paragraph{}
Trifft ein Team einen noch nicht geleerten Becher, der in einem vorherigen Zug getroffen wurde, so hat es das Spiel sofort gewonnen.
Dies stellt ein alternatives Spielende zu dem in~\ref{Bier-Pong:Spielablauf} geschilderten dar.
Es gibt keinen Nachwurf.

\subsection{Zusammenstellen}
\paragraph{}
Hat ein Team zu Beginn eines Zuges noch einen, drei oder sechs Becher zu treffen, so darf es die Becher zusammenstellen lassen.
Die Becher werden dabei wieder zu einer Pyramiden-Formation aufgestellt.
Um herunterfallende Becher zu vermeiden ist wieder ein kleiner Abstand zur Tischkante einzuhalten.

\paragraph{}
Fordert ein Team zu einem anderen Zeitpunkt als zu Beginn des eigenen Zuges das Zusammenstellen, so muss das gegnerische Team dem Wunsch weder direkt noch unaufgefordert zum angemessenen Zeitpunkt nachkommen.

\paragraph{}
Verrutschte Becher dürfen zwischen Zügen wieder gerade gerückt werden.

\subsection{Besondere Situationen}
\paragraph{}
\glqq{} Balls Back\grqq{}: Trifft ein Team in einem Zug mit beiden Bällen, so kommt es direkt noch einmal zum Zug.
In diesem Fall hat es aber keine Möglichkeit die Becher zusammenzustellen.

\paragraph{}
Trifft ein Team in einem Zug mit beiden Bällen in denselben Becher, müssen zwei zusätzliche Becher getrunken werden.
Die Becher werden vom werfenden Team frei gewählt.

\paragraph{}
\glqq{} 1UP\grqq{}: Rollt ein geworfener Ball auf dem Tisch zurück, so darf das werfende Team ihn wieder an sich nehmen, sobald er die Mittellinie überschritten hat.
Der Ball darf in Form eines Trickshots erneut geworfen werden.
Im Falle eines Treffers muss aber kein zusätzlicher Becher getrunken werden.

\paragraph{}
Wird auf einem Tisch mit Löchern für die Becher gespielt, so kann es dazu kommen, dass der Ball ohne eine einzige Berührung durch eines der Löcher geworfen wird.
Dieser Wurf zählt nicht als normaler Airball, sondern, als hätte jedes Teammitglied des verteidigenden Teams einen Airball geworfen.

\paragraph{}
Kippt ein Becher durch den Wurf um, so wird dieser direkt wieder aufgestellt und aufgefüllt.
Er gilt nicht als getroffen.

\section{Flunkyball}
\subsection{Aussprache}
\paragraph{}
Flunkyball wird \textipa{["fl\textturnv Nki\textsecstress b\textopeno l]} ausgesprochen.

\paragraph{}
Die Aussprache \textipa{["flUnki\textsecstress bal]} ist möglich, sollte aber vermieden werden.

\subsection{Vorbereitung}
\paragraph{}
Die Spielenden teilen sich in zwei etwa gleich große Teams auf.
Alle Spielenden erhalten eine 0,33l Flasche Bier.

\paragraph{}Variationen der Getränke entsprechend~\ref{Allgemeine_Regelungen:Getränke} sind nur nach Absprache mit dem gegnerischen Team möglich.

Eine Ausnahme bildet hierbei Radler.
Es ist auch ohne Absprache möglich 0,33l Bier gegen 0,33l Radler zu tauschen.

\paragraph{}
Spielende können mit 0,5l Flaschen spielen. Der dadurch entstehende Nachteil wird nicht ausgeglichen.

\paragraph{}
Die Teams stehen sich in zwei parallelen Reihen im Abstand von zehn bis 15 Schritten gegenüber.
Entlang der Reihen verteilen sich die Teammitglieder gleichmäßig und stellen ihre Flaschen auf den Boden.
In der Mitte steht stabil eine zu etwa 25\% mit Wasser gefüllte Plastikflasche (die Zielflasche).

\paragraph{}
Ein Wurfobjekt wird bereitgestellt.
Je nach Gewicht des Wurfobjekts kann die Wassermenge in der Zielflasche variiert werden um ein optimales Verhältnis zu erhalten.
Übliche Wurfobjekte umfassen Handbälle, mindestens halb volle Wasserflaschen und Frisbees.

\subsection{Spielablauf}
\paragraph{}
Ein zufälliges Team beginnt mit seinem Wurf.
Anschließend wirft abwechselnd je ein Teammitglied aus beiden Teams.

\paragraph{}
Wenn nicht anderweitig beschrieben, dürfen die Spielenden die Linie auf denen ihre Biere stehen nicht übertreten und ihre Bierflaschen nicht berühren.

\paragraph{}
Trifft ein Team bei seinem Wurf die Zielflasche und wirft diese um, so beginnt seine Trinkphase.
Sie endet, wenn das gegnerische Team die Flasche etwa am selben Ort wieder aufgerichtet hat, sich wieder vollständig und mitsamt dem Wurfobjekts hinter der eigenen Reihe der Bierflaschen befindet und deutlich hörbar \glqq{} Stopp\grqq{} gerufen hat.
Während der Trinkphase dürfen die Mitglieder des entsprechenden Teams ihre Flaschen berühren und auch daraus trinken.

\paragraph{}
Um aus dem Spiel auszuscheiden, halten Spielende ihre Flaschen für einige Sekunden mit der Öffnung nach unten.
Fließt nichts aus der Flasche, scheiden die Spielenden aus.
Sind alle Teammitglieder eines Teams erfolgreich ausgeschieden, hat das Team gewonnen.
Bei einzelnen Tropfen und sehr geringen Mengen Schaum sind die Spielenden des gegnerischen Teams angehalten etwas nachsichtig zu sein.

\subsection{Das Werfen}
\paragraph{}
Geworfen wird von hinter der Bierlinie des eigenen Teams von unten.
Ausnahme bilden Wurfobjekte, die eine abweichende Wurftechnik vorschreiben.

\paragraph{}
Geworfen wird primär auf die Zielflasche.
Es ist Spielenden allerdings auch erlaubt auf die Bierflaschen des gegnerischen Teams zu zielen.

\paragraph{}
Innerhalb der Teams werfen alle Teammitglieder der Reihe nach, sodass alle Teilnehmenden etwa gleich viel werfen.
Ausgeschieden Teammitglieder werfen nicht mehr.

\paragraph{}
Gastwürfe von nicht spielenden Personen sind erlaubt, wenn das gegnerische Team sein Einverständnis gibt.

\subsection{Verteidigen}
\paragraph{}
Unter Verteidigen versteht sich das Schützen der eigenen Bierflasche oder der Bierflasche eines Teammitglieds davor umgeworfen zu werden.

\paragraph{}
Verteidigt werden darf ab dem Moment, in dem das Wurfobjekt die Hand der Werfer:in verlässt.
Zu diesem Zweck darf die Bierlinie überschritten werden.
Mindestens ein Fuß muss aber zu jeder Zeit hinter der Bierlinie auf dem Boden bleiben.

\paragraph{}
Die eigene Bierflasche oder die Bierflasche eines Teammitglieds darf zum Verteidigen nicht berührt werden.

\paragraph{}
Die Bierlinie darf erst zum Zwecke des Aufstellens der Zielflasche und zum Wiedererlangen des Wurfobjekts vollends überschritten werden, sobald die Zielflasche getroffen wurde.

\subsection{Strafbier}
\paragraph{}\label{Flunkyball:Strafbier:Allg}
Wird Bier aus der eigenen Flasche verschüttet, so erhält man ein Strafbier.
Das bisherige Bier ist zügig zu leeren und wird anschließend durch eine neue, volle Flasche ersetzt.

\paragraph{}
Unter~\ref{Flunkyball:Strafbier:Allg} zählt explizit auch Bierschaum.

\paragraph{}
Unter~\ref{Flunkyball:Strafbier:Allg} zählt explizit auch verschüttetes Bier, wenn die Flasche im mit dem Ziel des Ausscheidens umgedreht wird.

\paragraph{}
Werden Regelbrüche festgestellt, wird die Person zunächst verwarnt.
Bei wiederholtem Regelbruch erhält sie ein Strafbier.

\chapter{Rage Cage / Kalaschnikow}
\section{Vorbereitung}
\begin{enumerate}[label={(\arabic*)}]
    \item
    Pro Spieler:in werden je \SI{0,33}{\liter} Bier auf drei Becher verteilt.
    Diese werden in einer zwei Becher breiten Reihe in der Tischmitte aufgestellt.
    Die Spielenden verteilen sich gleichmäßig um den Tisch.
    Zwei beliebige, sich gegenüberstehende Spieler:innen erhalten je einen leeren Becher und einen Tischtennisball.
    Sie sind die zwei Startspieler:innen.

    \item
    Variationen der Getränke sind entsprechend~\ref{Allgemeine_Regelungen:Getränke} möglich.
\end{enumerate}

\section{Spielablauf}
\begin{enumerate}[label={(\arabic*)}]
    \item
    Auf ein gemeinsames Kommando beginnen die Startspieler:innen zu spielen.
    Das Spiel endet, wenn alle Becher getrunken sind.

    \item
    Ist eine Spieler:in am Zug, so versucht sie ihren Tischtennisball indirekt in den vor ihr stehenden Becher zu werfen, indem sie ihn auf dem Tisch aufprallen lassen.
    Wird der Becher getroffen, so wird er im Normalfall an die Nachbar:in im Uhrzeigersinn weitergegeben.
    Die Nachbar:in ist nun am Zug.

    \item
    Trifft eine Spieler:in den Becher im ersten Versuch, so darf sie ihn an eine beliebige Mitspieler:in weitergeben.
\end{enumerate}

\section{Trinken}
\begin{enumerate}[label={(\arabic*)}]
    \item
    Trifft eine Spieler:in während die Nachbar:in, an die sie den Becher weitergeben müsste, am Zug ist, so stellt sie den getroffenen Becher samt Ball in den noch nicht getroffenen Becher der Nachbar:in.
    Diese nimmt sich einen der bereitstehenden Becher aus der Mitte, trinkt ihn aus und spielt anschließend mit diesem Becher weiter.
    Der entstandene Becherturm wird direkt nach dem Treffen noch einen Platz im Uhrzeigersinn weitergegeben und die entsprechende Spieler:in ist am Zug und darf sofort mit dem Werfen beginnen.

    \item
    Trifft eine Spieler:in einen der Becher in der Mitte anstatt ihren zu bespielenden Becher, so muss dieser getrunken und in den zu bespielenden Becher gestellt werden.
    Sie bleibt in diesem Fall am Zug.
\end{enumerate}

\section{Fairness}
\begin{enumerate}[label={(\arabic*)}]
    \item
    Spielenden steht es nach eigenen Treffern frei einen \glqq{}Ehrenwurf\grqq{} zu gewähren.
    Dies bedeutet, dass Trinkende mindestens einen Wurfversuch nach dem Trinken haben.

    \item
    Wird kein Ehrenwurf gewährt besteht die Möglichkeit, dass Spielende bereits wieder zum Trinken verpflichtet werden, bevor sie ihren aktuellen Becher fertig getrunken haben.
    In diesem Fall zählt der Treffer normal und die Spielenden trinken direkt einen weiteren Becher.
    Dies kann sich theoretisch bis zum Spielende ziehen.

    \item
    Stimmt die Mehrheit der Spielenden dafür, ist es für die Dauer des Spiels obligatorisch Ehrenwürfe zu gewähren.
\end{enumerate}

\section{Ring of Fire}
\subsection{Vorbereitung}
\paragraph{}
Drei bis sechs Personen spielen gegeneinander. Die Karten eines 52er Kartensets werden verdeckt in einem sich überlappenden Kreis in der Mitte des Tischs ausgebreitet. Alle Spielenden statten sich mit einem entsprechend §5 etwa gleichwertigen Getränk ihrer Wahl aus.

\paragraph{}
Trinkverpflichtungen werden in Einheiten von einzelnen Schlücken verteilt, wenn nicht explizit anderweitig spezifiziert.

\paragraph{}
Abweichungen von der Anzahl der Spielenden sind möglich, ändern jedoch die Spieldynamik nachhaltig. Anpassungen der Regeln, um ein zügigeres Fortschreiten zu ermöglichen, liegen in der Macht der Spielenden.

\paragraph{}
Ein 32er Kartenset eignet sich für Gruppen mit vier oder weniger Spielenden ebenfalls.

\subsection{Spielablauf}
\paragraph{}
\subparagraph{}
Early Game: Beginnend mit der Person zur Linken der mischenden Spieler:in ziehen die Spielenden gegen den Uhrzeigersinn nacheinander Karten aus dem Kreis. Gezogene Karten werden sofort offen gespielt. Die der Karte zugeordnete Aktion (siehe §15) tritt sofort in Kraft. Anschließend zieht die nächste Person eine Karte.
\subparagraph{}
Durchbruch: Ist der Kreis nach dem Zug einer Person zum ersten Mal durchbrochen, muss die entsprechende Person ihr Getränk exen.
\subparagraph{}
Late Game: Nach dem ersten Durchbruch des Kreises geht das Spiel wie im Early Game weiter. Die Spielenden müssen nun jedoch nicht mehr darauf achten, keine Lücken entstehen lassen, da es keine weiteren diesbezüglichen Strafen gibt.

\paragraph{}
Beim Ziehen einer Karte darf nur die gezogene Karte aktiv bewegt werden.

\paragraph{}
Zur aktiven Bewegung einer Karte zählt insbesondere das Greifen und Verschieben mit direkter Berührung. Indirektes Bewegen anderer Karten im Rahmen des Ziehens sind erlaubt.

\paragraph{}
Der Kreis gilt als durchbrochen, wenn der Tisch ununterbrochen von außerhalb bis innerhalb des Kreises sichtbar ist. Liegen Karten lediglich direkt nebeneinander ohne, dass eine Lücke entsteht, ist der Kreis nicht durchbrochen.

\subsection{Karten} \label{Ring_of_Fire:Karten}
\paragraph{}
Die Farben der Karten spielen keine Rolle. Lediglich die Wertigkeit ist entscheidend.

\paragraph{}
Die den Karten zugeordneten Aktionen sind wie folgt:

\begin{tabular}{p{1em} p{4em} p{23em}}
	2  & Two-You        & Der Mensch am Zug bestimmt, wer trinkt.                                                                                                              \\[1ex]
	3  & Three-Me       & Der Mensch am Zug trinkt.                                                                                                                            \\[1ex]
	4  & Four-Floor     & Die letzte Person, die ihre Hände unter dem Tisch hat, trinkt.
	Befindet sich die Ruheposition der Hände bei der Mehrheit der spielenden unterhalb der Tischkante, trinkt stattdessen die letzte Person, die ihre Hände auf dem Boden hat. \\[1ex]
	5  & Five-Guys      & Alle Männer trinken.                                                                                                                                 \\[1ex]
	6  & Six-Chicks     & Alle Frauen trinken.                                                                                                                                 \\[1ex]
	7  & Seven-Heaven   & Die letzte Person, die ihre Hände über den Kopf hebt, trinkt.                                                                                        \\[1ex]
	8  & Eight-Mate     & Die Person am Zug bestimmt eine Mitspieler:in zu ihrem Trink-Mate.                                                                                   \\[1ex]
	9  & NHIE           & Es wird \glqq Never Have I Ever\grqq{} gespielt, bis das erste Mal jemand trinkt.                                                                    \\[1ex]
	10 & Kategorie      & Die Person am Zug bestimmt eine Kategorie.
	Beginnend mit ihr selbst listen die Spielenden gegen den Uhrzeigersinn Elemente aus dieser Kategorie auf.
	Bei längeren Denkpausen können die Mitspielenden einen fünf-sekündigen Countdown starten.
	Läuft ein Countdown aus, wird ein Element doppelt genannt oder gibt eine Person auf, verliert sie und muss trinken.                                                        \\[1ex]
	B  & Thumb Master   & Die Person am Zug ist Thumb Master, bis der nächste Bube aufgedeckt wird oder bis Spielende.                                                         \\[1ex]
	D  & Question Queen & Die Person am Zug ist Question Queen, bis die nächste Dame aufgedeckt wird oder bis Spielende.                                                       \\[1ex]
	K  & Regel          & Die Person am Zug bestimmt eine zusätzliche Regel, die bis zum Spielende gilt.                                                                       \\[1ex]
	A  & Wasserfall     & Aller Spielenden beginnen zeitgleich ihr Getränk zu leeren.
	Mit Ausnahme der Person am Zug dürfen sie erst absetzen, wenn ihr Getränk leer ist oder die Person zu ihrer Rechten absetzt.
	Es besteht keine Pflicht abzusetzen, wenn man darf und das Getränk noch nicht leer ist.
	Leert eine Person ihr Getränk, so scheidet sie schlicht aus dem Spiel aus.
	Für die verbleibenden Spielenden ergibt sich die Reihenfolge, in der sie absetzen dürfen, als hätte die Person nie teilgenommen.                                           \\[1ex]
\end{tabular}

\paragraph{}
Trink-Mates trinken immer, wenn einer der Mates trinken muss.
Müssen mehrere Mates trinken, wird für jeden Mate einzeln getrunken.

\paragraph{}\label{Ring_of_Fire:Mates}
Hat eine Spielende noch keine Mates, so steht ihr die Wahl frei.
Anderenfalls dürfen nur Spielende ohne Mates gewählt werden.
Ist dies nicht möglich, werden alle bestehenden Mate-Verbindungen aufgelöst und anschließend gewählt.

\paragraph{}
Stimmt die Mehrheit der Spielenden dafür, kann \ref{Ring_of_Fire:Mates} ausgesetzt werden.
In diesem Fall werden alle bestehenden Mate-Verbindungen aufgelöst, wenn eine 8 gezogen wurde und durch eine weitere Verbindung alle Spielenden bis auf maximal einen miteinander verbunden wären.

\paragraph{}
Thumb Master: Gibt es einen Thumb Master, darf dieser jederzeit den Daumen sichtbar auf der Tischkante platzieren.
Die letzte Person, die ihren Daumen ebenfalls auf der Tischkante platziert, trinkt.
Es ist verboten, den Daumen dauerhaft auf der Tischplatte ruhen zu lassen, wenn man nicht Thumb Master ist.
Verstoß wird mit Trinken bestraft.
Sind lediglich Personen aus einer zusammengehörigen Mate-Verbindung übrig, zählt die Runde als beendet und jede übrige Person trinkt.

\paragraph{}
Question Queen: Fragen der Question Queen dürfen nicht beantwortet werden.
Dazu zählen explizit auch nonverbale Antworten. Verstoß wird mit Trinken bestraft.
Die einzige Ausnahme besteht, wenn die Antwort mit den Worten \glqq Fuck You\grqq{} eingeleitet wird.

\subsection{Regeln}
\paragraph{}
Von Spielenden aufgestellte Regeln dürfen weder Grundregeln noch bereits geltenden von Spielenden aufgestellten Regeln direkt widersprechen.

\paragraph{}
Von Spielenden aufgestellte Regeln dürfen keine Subgruppen oder Einzelpersonen grundsätzlich benachteiligen.

\chapter{Pferderennen}
\section{Vorbereitung}
\begin{enumerate}[label={(\arabic*)}]    
    \item
    Die Asse eines 52er Kartensets werden zur Seite gelegt und die übrigen Karten gut gemischt.
    Sieben zufällige Karten werden verdeckt in einer Reihe auf den Tisch gelegt.
    Die Asse werden offen in einer Spalte orthogonal zu den verdeckten Karten auf den Tisch gelegt, sodass die Karten gemeinsam ein 4 \begin{math}\times\end{math} 7 Feld abstecken.
    Die verdeckten Karten bilden die Seitenlinie des Spielfeldes.
    \begin{verbatim}
        +---+---+----+----+----+----+----+----+
        |   | X |  X |  X |  X |  X |  X |  X |
        | A |   |    |    |    |    |    |    |
        | A |   |    |    |    |    |    |    |
        | A |   |    |    |    |    |    |    |
        | A |   |    |    |    |    |    |    |
        +---+---+----+----+----+----+----+----+
        
        A = Ass
        X = verdekte zufällige Karte
    \end{verbatim}

    \item
    Die Spieler:in setzen auf eines der vier Asse – die \glqq{}Pferde\grqq{}.
    Gesetzt werden Schlucke oder Shots, die direkt selbst getrunken werden.
    Wie viele Schlucke oder Shots gesetzt werden und auf welches Pferd ist den Spielenden überlassen.
\end{enumerate}

\section{Spielablauf}
\begin{enumerate}[label={(\arabic*)}]
    \item
    Nach und nach werden die übrigen Karten aufgedeckt.
    Das entsprechende Pferd mit derselben Farbe darf um einen Platz nach vorne ziehen.
    Im ersten Zug liegt die Karte also neben der ersten verdeckten Karte und nach acht Schritten hinter der letzten.

    \item
    Sind alle Pferde mindestens auf Höhe einer verdeckten Karte, so wird diese aufgedeckt und das entsprechende Pferd wandert einen Schritt zurück.

    \item
    Gewonnen haben die Spielenden, deren Pferd als erstes alle verdeckten Karten hinter sich lässt.

    \item
    Alle Sieger:innen dürfen die doppelte Zahl der gesetzten Schlucke oder Shots unter den Mitspielenden verteilen.
    Alle Verlierer:innen trinken erneut die Zahl ihrer gesetzten Schlucke oder Shots.
\end{enumerate}

\section{Busfahren (Einzel)}\label{Busfahren_solo}
\subsection{Vorbereitung}
\paragraph{}
Gespielt wird von einer einzelnen Spieler:in.
Im Spieljargon \glqq{} fährt die Person Bus\grqq{}.

\paragraph{}
Unterstützt wird sie von der Dealer:in.
Die Dealer:in hält ein durchgemischtes 32er Kartenset mit den Kartenrücken nach oben vor sich.
Die Dealer:in wird \glqq{} Busfahrer\grqq{} genannt.

\subsection{Spielablauf}
\paragraph{}
Die Aufgabe der Spieler:in ist es, die als nächstes gelegten Karten vorherzusagen.
Errät sie vier Karten in Folge richtig, gewinnt sie und das Spiel endet.

\paragraph{}
Für jede falsch vorhergesagten Karte trinkt die Spieler:in einen Schluck.

\paragraph{}
Die Karten müssen in entsprechend der folgenden Sequenz vorhergesagt werden.
Bei einer falsch vorhergesagten Karte beginnt die Sequenz von Neuem:
\subparagraph{}\label{Busfahren_solo:Spielablauf:Fragen:RS}
Rot oder Schwarz? – Ist die Karte rot (Karo, Herz) oder schwarz (Kreuz, Pik)?
\subparagraph{}
Höher oder tiefer? – Ist die Karte höher oder tiefer als die vorherige Karte?
Es kann auch geraten werden, dass die Karte gleich ist.
\subparagraph{}
Dazwischen oder draußen? – Liegt die Karte zwischen den beiden vorherigen Karten oder außerhalb.
Es kann auch geraten werden, dass sie auf einer der Grenzen liegt.
\subparagraph{}
Rot oder Schwarz? – Siehe~\ref{Busfahren_solo:Spielablauf:Fragen:RS}.

\paragraph{}
Die Karten zählen von der niedrigsten zur höchsten folgendermaßen: 7, 8, 9, 10, B, D, K, A

\section{Busfahren (Gruppe) / Pyramide}
\subsection{Vorbereitung}
\paragraph{}
Auf dem Tisch werden 15 Spielkarten aus einem 32er Kartenset in Pyramidenform verdeckt ausgelegt.

\paragraph{}
Die drei bis fünf Spielenden erhalten je drei Spielkarten verdeckt. Gespielt wird mit linearen Bestrafungen (siehe~\ref{Busfahren_group:Spielablauf:Bestrafung:linear}).

\paragraph{}
Bei mehr Spielenden werden nur 10 Spielkarten ausgelegt. Gespielt wird mit exponentiellen Bestrafungen (siehe~\ref{Busfahren_group:Spielablauf:Bestrafung:exponentiell}).

\paragraph{}
Die übrigen Karten werden verdeckt bei Seite gelegt.

\subsection{Spielablauf}
\paragraph{}
Alle Spielenden machen sich mit ihren Karten vertraut.
Hierzu dürfen sie sich alle eigenen Karten anschauen und nach Belieben sortieren.
Anschließend legen sie die Karten verdeckt nebeneinander vor sich auf den Tisch.

\paragraph{}
Die Karten der Pyramide werden nun nach und nach aufgedeckt.
Zunächst werden die Karten der längsten Seite aufgedeckt und von dort aus der Reihe nach bis zur einzelnen Karte an der Spitze.
Nach jeder aufgedeckten Karte wird kurz pausiert.
Spielende, die Karten mit demselben Symbol vor sich ausliegen haben dürfen diese nun aufdecken und entsprechend~\ref{Busfahren_group:Spielablauf:Bestrafung} Schlücke unter allen Spielenden verteilen.

\paragraph{}
Schlücke werden in der Reihenfolge verteilt, wie die Spielenden angekündigt haben zu verteilen.

\paragraph{}\label{Busfahren_group:Spielablauf:Bestrafung}
Aus der Reihe, in der die aufgedeckte Karte lag, ergibt sich wie viele Schlücke verteilt werden dürfen
\subparagraph{}\label{Busfahren_group:Spielablauf:Bestrafung:linear}
Im Spielmodus linearer Bestrafung darf in der ersten Reihe ein Schluck verteilt werden.
Mit jeder zusätzlichen Reihe darf ein Schuck mehr verteilt werden.

\subparagraph{}\label{Busfahren_group:Spielablauf:Bestrafung:exponentiell}
Im Spielmodus exponentieller Bestrafung darf in der ersten Reihe ein Schluck verteilt werden.
Mit jeder zusätzlichen Reihe dürfen doppelt so viele Schlücke verteilt werden.

\paragraph{}
Hat jemand mehrfach dieselbe Karte vor sich ausliegen, so darf die Person für jede Karte einzeln Schlücke verteilen.

\paragraph{}
Karten für die Schlücke verteilt wurden, werden abgelegt.

\paragraph{}\label{Busfahren_group:Spielablauf:Ziehen}
Deckt jemand eine falsche Karte auf, so trinkt die Person selbst so viele Schlucke wie sie hätte verteilen dürfen.
Die Karte ist anschließend aus dem Spiel und die Person zieht eine neue zufällige Karte aus den nicht verteilten Karten.
Sind diese aufgebraucht wird zufällig aus den abgelegten Karten gezogen.

\paragraph{}
Die Person, die am Ende die meisten Karten auf der Hand hat, fährt Bus entsprechend~\ref{Busfahren_solo}.
Haben mehrere Personen gleich viele Karten auf der Hand wird durch Sudden Death entschieden, wer Bus fährt.
Im Sudden Death sagen die Spielenden abwechselnd die Farbe von Karten vorher.
Die erste Person, die einen Fehler macht, während die andere Person richtig liegt, fährt Bus.
Die jüngere Person beginnt mit dem Raten.

\section{Flunkern}
\paragraph{}
Vor Spielbeginn können die Spielenden sich darauf einigen, dass mit Flunkern gespielt wird.

\paragraph{}
In diesem Spielmodus werden Karten nicht direkt aufgedeckt, wenn man behauptet mit ihnen Schlücke verteilen zu dürfen.
Wer verpflichtet wird zu trinken darf die Behauptung anzweifeln.
Die entsprechenden Karten werden aufgedeckt und entsprechend~\ref{Busfahren_group:Spielablauf:Ziehen} durch neue Karten ersetzt.

\paragraph{}
Wird eine Karte zurecht angezweifelt, muss die Lügner:in die entsprechenden Schlücke selbst trinken.
Wird eine Karte fälschlicherweise angezweifelt, muss die anzweifelnde Person die doppelte Zahl an Schlücken trinken.

\paragraph{}
Im Spielmodus mit Flunkern wird nicht entsprechend~\ref{Busfahren_solo} Bus gefahren.

\chapter{Dreiermensch}

\section{Vorbereitung}
\begin{enumerate}[label={(\arabic*)}]	
	\item
	Alle Spielenden statten sich mit einem entsprechend~\ref{Allgemeine_Regelungen:Getränke} etwa gleichwertigen Getränk ihrer Wahl aus.
	Auf viele Spiele gerechnet, verhält sich die zu trinkende Menge pro Spieler im Schnitt proportional zur Anzahl der Spielenden.
	Der Dreiermensch jeder Runde trinkt dabei signifikant mehr als die restlichen Spielenden.
	Deshalb wird empfohlen, Dreiermensch mit maximal 6--8 Spielenden zu spielen.
	
	\item
	Es werden außerdem zwei faire sechsseitige Würfel benötigt.
\end{enumerate}

\section{Spielablauf}
\begin{enumerate}[label={(\arabic*)}]
	\item\label{Dreiermensch:Spielablauf:NeueRunde}
	Zu Beginn einer neuen Runde wirft jede Person reihum einen Würfel.
	Die Person, die die geringste Augenzahl gewürfelt hat, ist neuer Dreiermensch.
	Im Falle eines Gleichstandes, wird ein Stechen zwischen den am Gleichstand beteiligten Spielern ausgewürfelt, bis ein eindeutiger Dreiermensch bestimmt ist.
	
	\item
	Die Person, die in Spielrichtung neben dem Dreiermensch sitzt, beginnt die Runde mit ihrem Zug (\glqq{}aktive Person\grqq{}).
	
	\item
	Die aktive Person würfelt mit beiden Würfeln so, dass alle Beteiligten das Ergebnis zeitgleich sehen können.
	Führt der Wurf zu einem Ergebnis, bei dem mindestens eine Person trinken muss, werden zunächst die entsprechenden Schlücke getrunken.
	Die aktive Person ist anschließend erneut am Zug.
	Anderenfalls ist die nächste Person in Spielrichtung am Zug.
	
	\item
	Bei den folgenden Ergebnissen muss getrunken werden:

	%TODO this table is for whatever reason offset... Latex so much fun
	\begin{tabulary}{0.92\textwidth}{llL}
		\toprule
		Würfelergebnis   & Eselsbrücke & Aktion \\
		2 und 1          & Mäxchen & Alle müssen einen Schluck trinken \\
		Pasch (x und x)  &         & Die aktive Person darf x Schlucke unter den Mitspielenden aufteilen \\
		Summe = 7        & Heaven  & siehe~\ref{Ring_of_Fire:Karten:Tabelle} \\
		Summe = 11       & Heaven  & siehe~\ref{Ring_of_Fire:Karten:Tabelle} \\
		Summe = 4        & Floor   & siehe~\ref{Ring_of_Fire:Karten:Tabelle} \\
		Für jede 3 &         &Der Dreiermensch trinkt einen Schluck \\\bottomrule
	\end{tabulary}

	\item
	Ist der Dreiermensch die aktive Person und würfelt selbst eine 3, so endet die Runde, nachdem der Trinkschuld dieses letzten Zugs nachgekommen wurde.
	Anschließend wird entweder eine neue Runde nach~\ref{Dreiermensch:Spielablauf:NeueRunde} gestartet oder das Spiel beendet.
	
	\item
	Verlässt eine Person temporär das Spiel, zum Beispiel zum Zwecke der Beschaffung eines neuen Getränks oder des Besuchs der Toilette, so werden die Schlucke für diese Person mitgezählt und müssen bei Wiedererscheinen nachgetrunken werden.
	
	\item
	Es steht den Spielenden frei vor Beginn des Spieles festzulegen, wie viele Runden gespielt werden.
	Außerdem kann eine Obergrenze bestimmt werden, wie häufig jede Person Dreiermensch sein kann.
	Wird diese Grenze für eine Person erreicht, so nimmt sie am Auswürfeln nach~\ref{Dreiermensch:Spielablauf:NeueRunde} nicht teil.
\end{enumerate}

\chapter{Buffalo}
\section{Geltung}
\begin{enumerate}[label={(\arabic*)}]
    \item
    Um zu gelten, muss die Buffalo-Regel explizit an dem betreffenden Abend ausgerufen werden.

    \item
    Wurde die Buffalo-Regel im Rahmen eines Spiels als Regel eingeführt, gilt sie automatisch nur bis zum Ende des entsprechenden Spiels.
\end{enumerate}

\section{Die Regel}
\begin{enumerate}[label={(\arabic*)}]
    \item
    Es darf nur mit der nicht-dominanten Hand getrunken werden.

    \item
    Wer eine andere Person beim Regelbruch ertappt, erklärt dies mit dem Ausruf „Buffalo“.
    Die ertappte Person muss ihr Getränk exen.

    \item
    Regelbrüche müssen direkt erkannt und ausgerufen werden.
    Die Möglichkeit der Bestrafung erlischt, wenn das Getränk wieder in Ruheposition angelangt ist.

    \item
    Die Bestrafung des Regelbruchs kann durch die Personen in unmittelbarer Umgebung herabgesetzt werden.
\end{enumerate}


\end{document}

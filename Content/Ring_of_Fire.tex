\section{Ring of Fire}
\subsection{Vorbereitung}
\paragraph{}
Drei bis sechs Personen spielen gegeneinander. Die Karten eines 52er Kartensets werden verdeckt in einem sich überlappenden Kreis in der Mitte des Tischs ausgebreitet. Alle Spielenden statten sich mit einem entsprechend §5 etwa gleichwertigen Getränk ihrer Wahl aus.

\paragraph{}
Trinkverpflichtungen werden in Einheiten von einzelnen Schlücken verteilt, wenn nicht explizit anderweitig spezifiziert.

\paragraph{}
Abweichungen von der Anzahl der Spielenden sind möglich, ändern jedoch die Spieldynamik nachhaltig. Anpassungen der Regeln, um ein zügigeres Fortschreiten zu ermöglichen, liegen in der Macht der Spielenden.

\paragraph{}
Ein 32er Kartenset eignet sich für Gruppen mit vier oder weniger Spielenden ebenfalls.

\subsection{Spielablauf}
\paragraph{}
\subparagraph{}
Early Game: Beginnend mit der Person zur Linken der mischenden Spieler:in ziehen die Spielenden gegen den Uhrzeigersinn nacheinander Karten aus dem Kreis. Gezogene Karten werden sofort offen gespielt. Die der Karte zugeordnete Aktion (siehe §15) tritt sofort in Kraft. Anschließend zieht die nächste Person eine Karte.
\subparagraph{}
Durchbruch: Ist der Kreis nach dem Zug einer Person zum ersten Mal durchbrochen, muss die entsprechende Person ihr Getränk exen.
\subparagraph{}
Late Game: Nach dem ersten Durchbruch des Kreises geht das Spiel wie im Early Game weiter. Die Spielenden müssen nun jedoch nicht mehr darauf achten, keine Lücken entstehen lassen, da es keine weiteren diesbezüglichen Strafen gibt.

\paragraph{}
Beim Ziehen einer Karte darf nur die gezogene Karte aktiv bewegt werden.

\paragraph{}
Zur aktiven Bewegung einer Karte zählt insbesondere das Greifen und Verschieben mit direkter Berührung. Indirektes Bewegen anderer Karten im Rahmen des Ziehens sind erlaubt.

\paragraph{}
Der Kreis gilt als durchbrochen, wenn der Tisch ununterbrochen von außerhalb bis innerhalb des Kreises sichtbar ist. Liegen Karten lediglich direkt nebeneinander ohne, dass eine Lücke entsteht, ist der Kreis nicht durchbrochen.

\subsection{Karten}\label{Ring_of_Fire:Karten}
\paragraph{}
Die Farben der Karten spielen keine Rolle. Lediglich die Wertigkeit ist entscheidend.

\paragraph{}
Die den Karten zugeordneten Aktionen sind wie folgt:

\begin{tabular}{p{1em} p{4em} p{23em}}
	2  & Two-You        & Der Mensch am Zug bestimmt, wer trinkt.                                                                                                              \\[1ex]
	3  & Three-Me       & Der Mensch am Zug trinkt.                                                                                                                            \\[1ex]
	4  & Four-Floor     & Die letzte Person, die ihre Hände unter dem Tisch hat, trinkt.
	Befindet sich die Ruheposition der Hände bei der Mehrheit der spielenden unterhalb der Tischkante, trinkt stattdessen die letzte Person, die ihre Hände auf dem Boden hat. \\[1ex]
	5  & Five-Guys      & Alle Männer trinken.                                                                                                                                 \\[1ex]
	6  & Six-Chicks     & Alle Frauen trinken.                                                                                                                                 \\[1ex]
	7  & Seven-Heaven   & Die letzte Person, die ihre Hände über den Kopf hebt, trinkt.                                                                                        \\[1ex]
	8  & Eight-Mate     & Die Person am Zug bestimmt eine Mitspieler:in zu ihrem Trink-Mate.                                                                                   \\[1ex]
	9  & NHIE           & Es wird \glqq{} Never Have I Ever\grqq{} gespielt, bis das erste Mal jemand trinkt.                                                                    \\[1ex]
	10 & Kategorie      & Die Person am Zug bestimmt eine Kategorie.
	Beginnend mit ihr selbst listen die Spielenden gegen den Uhrzeigersinn Elemente aus dieser Kategorie auf.
	Bei längeren Denkpausen können die Mitspielenden einen fünf-sekündigen Countdown starten.
	Läuft ein Countdown aus, wird ein Element doppelt genannt oder gibt eine Person auf, verliert sie und muss trinken.                                                        \\[1ex]
	B  & Thumb Master   & Die Person am Zug ist Thumb Master, bis der nächste Bube aufgedeckt wird oder bis Spielende.                                                         \\[1ex]
	D  & Question Queen & Die Person am Zug ist Question Queen, bis die nächste Dame aufgedeckt wird oder bis Spielende.                                                       \\[1ex]
	K  & Regel          & Die Person am Zug bestimmt eine zusätzliche Regel, die bis zum Spielende gilt.                                                                       \\[1ex]
	A  & Wasserfall     & Aller Spielenden beginnen zeitgleich ihr Getränk zu leeren.
	Mit Ausnahme der Person am Zug dürfen sie erst absetzen, wenn ihr Getränk leer ist oder die Person zu ihrer Rechten absetzt.
	Es besteht keine Pflicht abzusetzen, wenn man darf und das Getränk noch nicht leer ist.
	Leert eine Person ihr Getränk, so scheidet sie schlicht aus dem Spiel aus.
	Für die verbleibenden Spielenden ergibt sich die Reihenfolge, in der sie absetzen dürfen, als hätte die Person nie teilgenommen.                                           \\[1ex]
\end{tabular}

\paragraph{}
Trink-Mates trinken immer, wenn einer der Mates trinken muss.
Müssen mehrere Mates trinken, wird für jeden Mate einzeln getrunken.

\paragraph{}\label{Ring_of_Fire:Mates}
Hat eine Spielende noch keine Mates, so steht ihr die Wahl frei.
Anderenfalls dürfen nur Spielende ohne Mates gewählt werden.
Ist dies nicht möglich, werden alle bestehenden Mate-Verbindungen aufgelöst und anschließend gewählt.

\paragraph{}
Stimmt die Mehrheit der Spielenden dafür, kann~\ref{Ring_of_Fire:Mates} ausgesetzt werden.
In diesem Fall werden alle bestehenden Mate-Verbindungen aufgelöst, wenn eine 8 gezogen wurde und durch eine weitere Verbindung alle Spielenden bis auf maximal einen miteinander verbunden wären.

\paragraph{}
Thumb Master: Gibt es einen Thumb Master, darf dieser jederzeit den Daumen sichtbar auf der Tischkante platzieren.
Die letzte Person, die ihren Daumen ebenfalls auf der Tischkante platziert, trinkt.
Es ist verboten, den Daumen dauerhaft auf der Tischplatte ruhen zu lassen, wenn man nicht Thumb Master ist.
Verstoß wird mit Trinken bestraft.
Sind lediglich Personen aus einer zusammengehörigen Mate-Verbindung übrig, zählt die Runde als beendet und jede übrige Person trinkt.

\paragraph{}
Question Queen: Fragen der Question Queen dürfen nicht beantwortet werden.
Dazu zählen explizit auch nonverbale Antworten. Verstoß wird mit Trinken bestraft.
Die einzige Ausnahme besteht, wenn die Antwort mit den Worten \glqq{} Fuck You\grqq{} eingeleitet wird.

\subsection{Regeln}
\paragraph{}
Von Spielenden aufgestellte Regeln dürfen weder Grundregeln noch bereits geltenden von Spielenden aufgestellten Regeln direkt widersprechen.

\paragraph{}
Von Spielenden aufgestellte Regeln dürfen keine Subgruppen oder Einzelpersonen grundsätzlich benachteiligen.

\section{Allgemeine Regelungen}\label{Allgemeine_Regelungen}

\subsection{Ehre}\label{Allgemeine_Regelungen:Ehre}
\paragraph{} Allgemein, aber insbesondere wo keine neutralen Beobachtenden beteiligt sind, verhalten sich die Spielenden als Ehrenmensch.
Es werden keine Regelverletzungen bewusst begangen oder fälschlich angeprangert oder abgestritten.

\paragraph{}\label{Allgemeine_Regelungen:Ehre:Spirit}
Um den \glqq{} Spirit of the Game\grqq{} zu erhalten können Spielende gemeinsam entscheiden Regeln auszusetzen.
Ausgenommen hiervon sind~\ref{Allgemeine_Regelungen:Ehre} und~\ref{Allgemeine_Regelungen:Gender}.

\paragraph{}
~\ref{Allgemeine_Regelungen:Ehre:Spirit} findet insbesondere Verwendung, wenn Personen am Spiel beteiligt sind, die grundsätzlich keinen Alkohol konsumieren möchten.
Diese sollen, wo möglich, den nicht ausgeschlossen werden.

\subsection{Gender}\label{Allgemeine_Regelungen:Gender}
\paragraph{}
Wird in einer Regel zwischen Männern und Frauen differenziert ist das Geschlecht gemeint, dem sich die Spielenden zum Zeitpunkt des Spiels zugehörig fühlen.

\paragraph{}
Menschen, die sich weder Männern noch Frauen zugehörig fühlen, wählen frei, mit welcher Gruppe sie für die Dauer des Spiels gemeinsam trinken.
Diese Wahl ist in keiner Weise zu interpretieren.

\subsection{Das Trinken}
\paragraph{}\label{Allgemeine_Regelungen:Trinken:Trink}
Wer spielt, trinkt.

\paragraph{}
~\ref{Allgemeine_Regelungen:Trinken:Trink} gilt nicht mehr, wenn sich eine Person entscheidet mit dem Trinken aufzuhören.
In diesem Fall kann die Spieler:in ausscheiden oder es wird entweder das Spiel abgebrochen, die Spieler:in ausgetauscht, ein Aussetzen der Trinkpflicht für diese Person entschieden oder die Trinkpflicht von einer freiwilligen Person übernommen.
Die Entscheidung treffen die Spielenden gemeinsam.

\paragraph{}
Spielende können jederzeit aus einem Spiel ausscheiden. Wenn für das Spiel nötig, sollten sie eine Ersatzspieler:in anwerben.

\subsection{Getränke}\label{Allgemeine_Regelungen:Getränke}
\paragraph{}\label{Allgemeine_Regelungen:Getränke:Umrechnung}
Es folgt ein allgemeiner Umrechnungsschlüssel für Getränke.
Wenn Spielende ein Getränk durch ein anderes ersetzen möchten, so können sie dies entsprechend den folgenden Verhältnissen tun.
1l Bier entspricht:

\begin{tabular}{l l}
	1l   & Apfelwein \\
	0,4l & Wein      \\
	0,3l & Pfeffi    \\
	2l   & Radler
\end{tabular}

\paragraph{}\label{Allgemeine_Regelungen:Getränke:Abfüllen}
Werden bereits vor Spielbeginn Trinkeinheiten abgefüllt, so werden diese anteilig nach den Wünschen der Spielenden bereitgestellt.
Ist eine Partei in der Trinkpflicht und hat bereits alle Anteile des eigenen Wunschgetränks aufgebraucht, so beginnt sie die übrigen Anteile nach eigener Wahl zu leeren.

\paragraph{}
Ausnahmen zu~\ref{Allgemeine_Regelungen:Getränke:Abfüllen} wie etwa ersetzten der Trinkeinheiten oder Tauschen mit den Mitspielenden sind möglich.

\paragraph{}
Muss eine Flasche Bier geext werden, steht es der betreffenden Person frei dies in Form eines Tornados zu tun, sofern es die Örtlichkeiten zulassen.
Solange der Versuch ernsthaft durchgeführt und ein Großteil der Flasche geleert wurde, gilt die Trinkpflicht als erfüllt auch, wenn noch Bier in der Flasche ist.

\paragraph{}
Wenn nicht anders spezifiziert beziehen sich Schluckangaben auf Bier.
Spielende mit einem anderen Getränk müssen entsprechend~\ref{Allgemeine_Regelungen:Getränke:Umrechnung} umrechnen.

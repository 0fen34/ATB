\chapter{Allgemeine Regelungen}\label{Allgemeine_Regelungen}

\section{Ehre}\label{Allgemeine_Regelungen:Ehre}
\begin{enumerate}[label={(\arabic*)}]
	\item Allgemein, aber insbesondere wo keine neutralen Beobachtenden beteiligt sind, verhalten sich die Spielenden als Ehrenmensch.
	Es werden keine Regelverletzungen bewusst begangen oder fälschlich angeprangert oder abgestritten.
	
	\item\label{Allgemeine_Regelungen:Ehre:Spirit}
	Um den Spirit of the Game\grqq{} zu erhalten können Spielende gemeinsam entscheiden Regeln auszusetzen.
	Ausgenommen hiervon sind~\ref{Allgemeine_Regelungen:Ehre} und~\ref{Allgemeine_Regelungen:Gender}.
	
	\item
~\ref{Allgemeine_Regelungen:Ehre:Spirit} findet insbesondere Anwendung, wenn Personen am Spiel beteiligt sind, die grundsätzlich keinen Alkohol konsumieren möchten.
	Diese sollen, wo möglich, nicht ausgeschlossen werden.
\end{enumerate}

\section{Gender}\label{Allgemeine_Regelungen:Gender}
\begin{enumerate}[label={(\arabic*)}]	
	\item
	Wird in einer Regel zwischen Männern und Frauen differenziert ist das Geschlecht gemeint, dem sich die Spielenden zum Zeitpunkt des Spiels zugehörig fühlen.
	
	\item
	Menschen, die sich weder Männern noch Frauen zugehörig fühlen, wählen frei, mit welcher Gruppe sie für die Dauer des Spiels gemeinsam trinken.
	Diese Wahl ist in keiner Weise zu interpretieren.
\end{enumerate}
	
\section{Das Trinken}
\begin{enumerate}[label={(\arabic*)}]
	\item\label{Allgemeine_Regelungen:Trinken:Trink}
	Wer spielt, trinkt.
	
	\item
~\ref{Allgemeine_Regelungen:Trinken:Trink} gilt nicht mehr, wenn sich eine Person für den Abend entscheidet mit dem Trinken aufzuhören.
	In diesem Fall kann die Spieler:in entweder ausscheiden, das Spiel wird abgebrochen, die Spieler:in wird ausgetauscht,
	ein Aussetzen der Trinkpflicht für diese Person wird entschieden oder die Trinkpflicht wird von einer freiwilligen Person übernommen.
	Die Entscheidung treffen die Spielenden gemeinsam.
	
	\item
	Spielende können nur aus dringenden Gründen aus einem Spiel ausscheiden.
	Wenn für das Spiel nötig, sollten sie eine Ersatzspieler:in anwerben.
\end{enumerate}

\section{Getränke}\label{Allgemeine_Regelungen:Getränke}
\begin{enumerate}[label={(\arabic*)}]
	\item\label{Allgemeine_Regelungen:Getränke:Umrechnung}
	Es folgt ein allgemeiner Umrechnungsschlüssel für Getränke.
	Wenn Spielende ein Getränk durch ein anderes ersetzen möchten, so können sie dies entsprechend den folgenden Verhältnissen tun.
	\SI{1}{\liter} Bier entspricht:
	\begin{tabular}{ll}
		\toprule
		Getränk   & Umrechnungsschlüssel \\
		Apfelwein & \SI{1}{\liter} \\
		Wein 	  & \SI{0,4}{\liter} \\
		Pfeffi 	  & \SI{0,3}{\liter} \\
		Radler    & \SI{2}{\liter} \\\bottomrule
	\end{tabular}
	
	\item\label{Allgemeine_Regelungen:Getränke:Abfüllen}
	Werden bereits vor Spielbeginn Trinkeinheiten abgefüllt, so werden diese anteilig nach den Wünschen der Spielenden bereitgestellt.
	Ist eine Partei in der Trinkpflicht und hat bereits alle Anteile des eigenen Wunschgetränks aufgebraucht, so beginnt sie die übrigen Anteile nach eigener Wahl zu leeren.
	
	\item
	Ausnahmen zu~\ref{Allgemeine_Regelungen:Getränke:Abfüllen} wie etwa ersetzten der Trinkeinheiten oder Tauschen mit den Mitspielenden sind möglich.
	
	\item
	Wenn nicht anders spezifiziert beziehen sich Schluckangaben auf Bier.
	Spielende mit einem anderen Getränk müssen entsprechend~\ref{Allgemeine_Regelungen:Getränke:Umrechnung} umrechnen.

	\item
	Werden in diesem Regelwerk spezifische Getränke referenziert, so beziehen sich die Regelungen selbstverständlich auch auf entsprechend~\ref{Allgemeine_Regelungen:Getränke:Umrechnung} variierte Getränke.
\end{enumerate}

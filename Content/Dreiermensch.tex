\chapter{Dreiermensch}

\section{Vorbereitung}
\begin{enumerate}[label={(\arabic*)}]	
	\item
	Alle Spielenden statten sich mit einem entsprechend~\ref{Allgemeine_Regelungen:Getränke} etwa gleichwertigen Getränk ihrer Wahl aus.
	Auf viele Spiele gerechnet, verhält sich die zu trinkende Menge pro Spieler im Schnitt proportional zur Anzahl der Spielenden.
	Der Dreiermensch jeder Runde trinkt dabei signifikant mehr als die restlichen Spielenden.
	Deshalb wird empfohlen, Dreiermensch mit maximal 6--8 Spielenden zu spielen.
	
	\item
	Es werden zwei faire sechsseitige Würfel bereitgestellt.
\end{enumerate}

\section{Spielablauf}
\begin{enumerate}[label={(\arabic*)}]
	\item\label{Dreiermensch:Spielablauf:NeueRunde}
	Zu Beginn einer neuen Runde wirft jede Person reihum einen Würfel.
	Die Person, die die geringste Augenzahl gewürfelt hat, ist neuer Dreiermensch.
	Im Falle eines Gleichstandes wird ein Stechen zwischen den am Gleichstand beteiligten Spielern ausgewürfelt, bis ein eindeutiger Dreiermensch bestimmt ist.
	
	\item
	Die Person, die in Spielrichtung neben dem Dreiermensch sitzt, beginnt die Runde mit ihrem Zug (\glqq{}aktive Person\grqq{}).
	
	\item
	Die aktive Person würfelt mit beiden Würfeln so, dass alle Beteiligten das Ergebnis zeitgleich sehen können.
	Führt der Wurf zu einem Ergebnis, bei dem mindestens eine Person trinken muss, werden zunächst die entsprechenden Schlucke getrunken.
	Die aktive Person ist anschließend erneut am Zug.
	Anderenfalls ist die nächste Person in Spielrichtung am Zug.
	
	\item
	Bei den folgenden Ergebnissen muss getrunken werden:

	%TODO this table is for whatever reason offset... Latex so much fun
	\begin{tabulary}{0.92\textwidth}{llL}
		\toprule
		Würfelergebnis   & Eselsbrücke & Aktion \\
		2 und 1          & Mäxchen & Alle müssen einen Schluck trinken \\
		Pasch (x und x)  &         & Die aktive Person darf x Schlucke unter den Mitspielenden aufteilen \\
		Summe = 7        & Heaven  & siehe~\ref{Ring_of_Fire:Karten:Tabelle} \\
		Summe = 11       & Heaven  & siehe~\ref{Ring_of_Fire:Karten:Tabelle} \\
		Summe = 4        & Floor   & siehe~\ref{Ring_of_Fire:Karten:Tabelle} \\
		Für jede 3 &         &Der Dreiermensch trinkt einen Schluck \\\bottomrule
	\end{tabulary}

	\item
	Ist der Dreiermensch die aktive Person und würfelt selbst eine 3, so endet die Runde, nachdem der Trinkschuld dieses letzten Zugs nachgekommen wurde.
	Anschließend wird entweder eine neue Runde nach~\ref{Dreiermensch:Spielablauf:NeueRunde} gestartet oder das Spiel beendet.
	
	\item
	Es steht den Spielenden frei vor Beginn des Spieles festzulegen, wie viele Runden gespielt werden.
	Außerdem kann eine Obergrenze bestimmt werden, wie häufig jede Person Dreiermensch sein kann.
	Wird diese Grenze für eine Person erreicht, so nimmt sie am Auswürfeln nach~\ref{Dreiermensch:Spielablauf:NeueRunde} nicht teil.
\end{enumerate}

\section{Dreiermensch}

\subsection{Vorbereitung}
\paragraph{}
Alle Spielenden statten sich mit einem entsprechend \ref{Allgemeine_Regelungen:Getränke} etwa gleichwertigen Getränk ihrer Wahl aus.
Auf viele Spiele gerechnet, verhält sich die zu trinkende Menge pro Spieler im Schnitt proportional zur Anzahl der Spielenden.
Der Dreiermensch jeder Runde trinkt dabei signifikant mehr als die restlichen Spielenden.
Deshalb wird empfohlen, Dreiermensch mit maximal 6-8 Spielenden zu spielen.

\paragraph{}
Es werden außerdem zwei faire sechsseitige Würfel benötigt.


\subsection{Spielablauf}
\paragraph{} \label{Dreiermensch:Spielablauf:NeueRunde}
Zu Beginn einer neuen Runde wirft jede Person reihum einen Würfel.
Die Person, die die geringste Augenzahl gewürfelt hat, ist neuer Dreiermensch.
Im Falle eines Gleichstandes, wird ein Stechen zwischen den am Gleichstand beteiligten Spielern ausgewürfelt, bis ein eindeutiger Dreiermensch bestimmt ist.

\paragraph{}
Die Person, die in Spielrichtung neben dem Dreiermensch sitzt, beginnt die Runde mit ihrem Zug (\glqq aktive Person\grqq{}).

\paragraph{}
Die aktive Person würfelt mit beiden Würfeln so, dass alle Beteiligten das Ergebnis zeitgleich sehen können.
Führt der Wurf zu einem Ergebnis, bei dem mindestens eine Person trinken muss, werden zunächst die entsprechenden Schlücke getrunken.
Die aktive Person ist anschließend erneut am Zug.
Anderenfalls ist die nächste Person in Spielrichtung am Zug.

\paragraph{}
Bei den folgenden Ergebnissen muss getrunken werden:

\begin{tabular}{p{8em} p{22em}}
     2 und 1: & Mäxchen – Alle müssen einen Schluck trinken\\
     Pasch (x und x): & Die aktive Person darf x Schlucke unter den Mitspielenden aufteilen\\
     Summe = 7: & Heaven – siehe \ref{Ring_of_Fire:Karten}\\
     Summe = 11: & Heaven – siehe \ref{Ring_of_Fire:Karten}\\
     Summe = 4: & Floor – siehe \ref{Ring_of_Fire:Karten}\\
     Pro geworfener 3: & Der Dreiermensch trinkt einen Schluck
\end{tabular}

\paragraph{}
Ist der Dreiermensch die aktive Person und würfelt selbst eine 3, so endet die Runde, nachdem der Trinkschuld dieses letzten Zugs nachgekommen wurde.
Anschließend wird entweder eine neue Runde nach \ref{Dreiermensch:Spielablauf:NeueRunde} gestartet oder das Spiel beendet.

\paragraph{}
Verlässt eine Person temporär das Spiel, zum Beispiel zum Zwecke der Beschaffung eines neuen Getränks oder des Besuchs der Toilette, so werden die Schlucke für diese Person mitgezählt und müssen bei Wiedererscheinen nachgetrunken werden.

\paragraph{}
Es steht den Spielenden frei vor Beginn des Spieles festzulegen, wie viele Runden gespielt werden.
Außerdem kann eine Obergrenze bestimmt werden, wie häufig jede Person Dreiermensch sein kann.
Wird diese Grenze für eine Person erreicht, so nimmt sie am Auswürfeln nach \ref{Dreiermensch:Spielablauf:NeueRunde} nicht teil.
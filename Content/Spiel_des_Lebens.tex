\chapter{Spiel des Lebens}
\section{Geltung}
\begin{enumerate}[label={(\arabic*)}]
    \item
    Um zu gelten, muss das Spiel des Lebens der betroffenen Person bekannt sein.

    \item
    Als Teil dieses Regelwerks ist das Spiel des Lebens per Definition allen Menschen bekannt, die über das Regelwerk und den Zugang dazu informiert wurden sowie ausreichend Zeit hatten, das Regelwerk zu lesen. 
\end{enumerate}

\section{Die Regel}
\begin{enumerate}[label={(\arabic*)}]
    \item
    Hat die Flasche, aus der getrunken wird, ein aufgeklebtes Etikett, so ist dieses vor dem Trinken zu beschädigen.

    \item
    Als beschädigt gilt ein Etikett, wenn es entweder eingerissen oder zum Teil von der Flasche abgetrennt wurde.

    \item
    Wer eine andere Person beim Regelbruch ertappt, erklärt dies mit der Frage \glqq{}Wie läuft's im Leben?\grqq{} oder einer Variation davon.
    Die ertappte Person muss ihr Getränk in einem Zug leeren.

    \item
    Regelbrüche müssen direkt erkannt und angemerkt werden.
    Die Möglichkeit der Bestrafung erlischt, wenn die Flasche den Kontakt zum Mund abbricht.

    \item
    Die Bestrafung des Regelbruchs kann durch die Personen in unmittelbarer Umgebung herabgesetzt werden.
\end{enumerate}

\chapter{Rage Cage / Kalaschnikow}
\section{Vorbereitung}
\begin{enumerate}[label={(\arabic*)}]
    \item
    Pro Spieler:in werden je 0,33l Bier auf drei Becher verteilt.
    Diese werden in einer zwei Becher breiten Reihe in der Tischmitte aufgestellt.
    Die Spielenden verteilen sich gleichmäßig um den Tisch.
    Zwei beliebige, sich gegenüberstehende Spieler:innen erhalten je einen leeren Becher und einen Tischtennisball.
    Sie sind die zwei Startspieler:innen.

    \item
    Variationen der Getränke sind entsprechend~\ref{Allgemeine_Regelungen:Getränke} möglich.
\end{enumerate}

\section{Spielablauf}
\begin{enumerate}[label={(\arabic*)}]
    \item
    Auf ein gemeinsames Kommando beginnen die Startspieler:innen zu spielen.
    Das Spiel endet, wenn alle Becher getrunken sind.

    \item
    Ist eine Spieler:in am Zug, so versucht sie ihren Tischtennisball indirekt in den vor ihr stehenden Becher zu werfen, indem sie ihn auf dem Tisch aufprallen lassen.
    Wird der Becher getroffen, so wird er im Normalfall an die Nachbar:in im Uhrzeigersinn weitergegeben.
    Die Nachbar:in ist nun am Zug.

    \item
    Trifft eine Spieler:in den Becher im ersten Versuch, so darf sie ihn an eine beliebige Mitspieler:in weitergeben.
\end{enumerate}

\section{Trinken}
\begin{enumerate}[label={(\arabic*)}]
    \item
    Trifft eine Spieler:in während die Nachbar:in, an die sie den Becher weitergeben müsste, am Zug ist, so stellt sie den getroffenen Becher samt Ball in den noch nicht getroffenen Becher der Nachbar:in.
    Diese nimmt sich einen der bereitstehenden Becher aus der Mitte, trinkt ihn aus und spielt anschließend mit diesem Becher weiter.
    Der entstandene Becherturm wird direkt nach dem Treffen noch einen Platz im Uhrzeigersinn weitergegeben und die entsprechende Spieler:in ist am Zug und darf sofort mit dem Werfen beginnen.

    \item
    Trifft eine Spieler:in einen der Becher in der Mitte anstatt ihren zu bespielenden Becher, so muss dieser getrunken und in den zu bespielenden Becher gestellt werden.
    Sie bleibt in diesem Fall am Zug.
\end{enumerate}

\section{Fairness}
\begin{enumerate}[label={(\arabic*)}]
    \item
    Die Spielenden können sich darauf einigen, dass Trinkenden ein \glqq{} Ehrenwurf\grqq{} gewährt wird.
    Dies bedeutet, dass sie mindestens einen Wurfversuch nach dem Trinken haben.

    \item
    Wird kein Ehrenwurf gewährt besteht die Möglichkeit, dass Spielende bereits wieder zum Trinken verpflichtet werden, bevor sie ihren aktuellen Becher fertig getrunken haben.
    In diesem Fall zählt der Treffer normal und die Spielenden trinken direkt einen weiteren Becher.
    Dies kann sich theoretisch bis zum Spielende ziehen.

    \item
    Spielenden steht es immer frei Ehrenwürfe nach ihren eigenen Treffern zu gewähren.

    \item
    Stimmt die Mehrheit der Spielenden dafür, ist es für die Dauer des Spiels obligatorisch Ehrenwürfe zu gewähren.
\end{enumerate}

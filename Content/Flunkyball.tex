\chapter{Flunkyball}
\section{Aussprache}
\begin{enumerate}[label={(\arabic*)}]
    \item
    Flunkyball wird \textipa{["fl\textturnv Nki\textsecstress b\textopeno l]} ausgesprochen.
\end{enumerate}

\section{Vorbereitung}
\begin{enumerate}[label={(\arabic*)}]
    \item
    Die Spielenden teilen sich in zwei etwa gleich große Teams auf.
    Die Größe der Teams darf zehn Personen nicht überschreiten.
    Zwischen den Teams darf die Größe um maximal eine Spieler:in abweichen.
    Alle Spielenden erhalten eine \SI{0,33}{\liter} Flasche Bier.

    \item Variationen der Getränke entsprechend~\ref{Allgemeine_Regelungen:Getränke} sind nur nach Absprache mit dem gegnerischen Team möglich.

    Eine Ausnahme bildet hierbei Radler.
    Es ist auch ohne Absprache möglich \SI{0,33}{\liter} Bier gegen \SI{0,33}{\liter} Radler zu tauschen.

    \item
    Spielende können freiwillig mit \SI{0,5}{\liter} Flaschen spielen.
    Der dadurch entstehende Nachteil wird nicht ausgeglichen.

    \item
    Die Teams stehen sich in zwei parallelen Reihen im Abstand von \SI{12}{\meter} gegenüber.
    Entlang der Reihen verteilen sich die Teammitglieder gleichmäßig, sodass ein Rechteck entsteht, und stellen ihre Flaschen auf den Boden.
    Die Kante des Rechtecks auf dem die Flaschen des eigenen Teams stehen, wird als \glqq{}Bierlinie\grqq{} bezeichnet.
    In der Mitte des Rechtecks steht stabil eine zu 50 \% mit Wasser gefüllte \SI{1}{\liter} PET-Flasche (die \glqq{}Zielflasche\grqq{}).
    Im Verlauf des Spiels darf die Zielflasche nie mehr als \SI{30}{\centi\meter} von dieser Ursprungsposition entfernt aufgestellt werden.

    \item
    Ein Wurfobjekt wird bereitgestellt.
    Je nach Gewicht des Wurfobjekts kann die Wassermenge in der Zielflasche variiert werden, um ein optimales Verhältnis zu erhalten.
    Übliche Wurfobjekte umfassen Handbälle, mindestens halb volle PET-Flaschen und Frisbees.
\end{enumerate}

\section{Spielablauf}
\begin{enumerate}[label={(\arabic*)}]    
    \item
    Ein zufälliges Team beginnt mit seinem Wurf.
    Das werfende Team wird als \glqq{}angreifend\grqq{} referenziert.
    Das gegenüberstehende Team als \glqq{}verteidigend\grqq{}.
    Die Teams werfen abwechselnd bis ein Team gewonnen hat.

    \item
    Wenn nicht explizit anderweitig beschrieben, dürfen die Spielenden die Linie auf denen ihre Biere stehen nicht übertreten.
    Selbiges gilt dafür, ihre Bierflaschen zu berühren.

    \item
    Trifft ein Team durch seinem Wurf die Zielflasche und wirft diese um, so beginnt seine Trinkphase.
    Sie endet, wenn das verteidigende Team die Flasche etwa am selben Ort wieder aufgerichtet hat, sich wieder vollständig und mitsamt dem Wurfobjekts hinter der eigenen Reihe der Bierflaschen befindet und deutlich hörbar \glqq{}Stopp\grqq{} gerufen hat.
    Während der Trinkphase dürfen die Mitglieder des angreifenden Teams ihre Flaschen berühren und auch daraus trinken.
    Nach Ende der Trinkphase darf kein Bier mehr getrunken werden.
    Die Flasche muss sofort in ihre Ausgangsposition am Boden.
    Dies darf in einer fließenden Bewegung passieren, während derer der Kontakt zum Mund bestehen bleiben darf.

    \item
    Um aus dem Spiel auszuscheiden, halten Spielende ihre Flaschen für einige Sekunden mit der Öffnung nach unten.
    Fließt nichts aus der Flasche, scheiden die Spielenden aus.
    Sind alle Teammitglieder eines Teams erfolgreich ausgeschieden, hat das Team gewonnen.
    Bei einzelnen Tropfen und sehr geringen Mengen Schaum sind die Spielenden des gegnerischen Teams angehalten, etwas nachsichtig zu sein.

    \item
    Um Ungenauigkeiten vorzubeugen, können Spielende fordern, die Kronkorkenregelung anzuwenden.
    In diesem Fall werden Flaschen zum Ausscheiden in einen Kronkorken entleert.
    Laufen weder Flüssigkeit noch Schaum über, scheidet die Spieler:in erfolgreich aus.
\end{enumerate}

\section{Das Werfen}
\begin{enumerate}[label={(\arabic*)}]
    \item
    Geworfen wird von hinter der Bierlinie des eigenen Teams von unten.
    Ausnahme bilden Wurfobjekte, die eine abweichende Wurftechnik vorschreiben.

    \item
    Geworfen wird erst nachdem das verteidigende Team bestätigt hat, bereit zu sein.

    \item
    Gezielt wird auf die Zielflasche.
    Absichtlich auf die Bierflaschen des gegnerischen Teams zu zielen, ist verboten.

    \item
    Unrechtmäßige Würfe, die etwa klar nicht auf die Zielflasche gerichtet waren oder ohne Bereitschaft des verteidigenden Teams geschehen sind, werden nicht wiederholt.

    \item
    Wird eine Bierflasche durch einen unrechtmäßigen Wurf umgekippt, so wird sie zügig wieder aufgestellt.
    Beim Umwerfen ausgetretener Inhalt hat hierbei keine Konsequenz.

    \item
    Innerhalb der Teams werfen alle Teammitglieder der Reihe nach, sodass alle Teilnehmenden etwa gleich viel werfen.
    Auch ausgeschiedene Teammitglieder dürfen werfen.

    \item
    Gastwürfe von nicht spielenden Personen sind erlaubt, wenn das gegnerische Team sein Einverständnis gibt.
\end{enumerate}

\section{Treffer}
\begin{enumerate}[label={(\arabic*)}]
    \item\label{Flunkyball:Treffer:Allg}
    Die Zielflasche gilt als getroffen, sobald sie mit einem anderen Teil als ihrem Boden den Untergrund berührt oder diesen vollständig verlässt.

    \item 
    Richtet sich die Zielflasche nach einem Treffer von alleine wieder auf und bleibt stabil stehen, wird verfahren als hätte das verteidigende Team die Flasche aufgestellt.
\end{enumerate}

\section{Verteidigen}
\begin{enumerate}[label={(\arabic*)}]
    \item
    Unter Verteidigen versteht sich das Schützen der eigenen Bierflasche oder der Bierflasche eines Teammitglieds davor umgeworfen zu werden.

    \item
    Verteidigt werden darf ab dem Moment, in dem das Wurfobjekt die Hand der Werfer:in verlässt.
    Zu diesem Zweck darf die Bierlinie mit maximal einem Fuß überschritten werden.

    \item
    Die eigene Bierflasche und die Bierflaschen der Teammitglieder dürfen zum Verteidigen nicht berührt werden.
    Eine Berührung ist erst erlaubt, nachdem die entsprechende Flasche umgekippt ist.
    Bierflaschen gelten hierbei unter den selben Umständen als umgekippt, unter denen die Zielflasche als getroffen gilt.

    \item
    Die Bierlinie darf ausschließlilch zum Zwecke des Aufstellens der Zielflasche und zum Wiedererlangen des Wurfobjekts vollständig überschritten werden.
    Dies darf erst geschehen, sobald die Zielflasche getroffen wurde.
\end{enumerate}

\section{Strafbier}
\begin{enumerate}[label={(\arabic*)}]
    \item\label{Flunkyball:Strafbier:Allg}
    Wird Bier aus der eigenen Flasche verschüttet, so erhält man ein Strafbier.
    Das bisherige Bier ist zügig zu leeren und wird anschließend durch eine neue, volle Flasche ersetzt.

    \item
    Unter~\ref{Flunkyball:Strafbier:Allg} zählt explizit auch Bierschaum, der in groben Maße übertritt.

    \item
    Unter~\ref{Flunkyball:Strafbier:Allg} zählt explizit auch verschüttetes Bier, wenn die Flasche im mit dem Ziel des Ausscheidens umgedreht wird.

    \item
    Wird Bier zu Unrecht getrunken, so erhält man ein Strafbier.

    \item
    Wird eine Flasche umgekippt, ohne dass Teil ihres Inhalts verschüttet wird, so ist kein Strafbier fällig.

    \item
    Werden Regelbrüche ohne klar definierte Strafe festgestellt, wird die Person zunächst verwarnt.
    Bei wiederholtem Regelbruch erhält sie ein Strafbier.

    \item
    Das bestrafte Team darf dem nicht bestraften Team alternativ zum Strafbier eine Strafsekunde vorschlagen.
    Wird diese akzeptiert, darf das nicht bestrafte Team insgesamt pro Mitglied für eine Sekunde trinken.
    Die Zeit ist möglichst gleichmäßig auf alle nicht ausgeschiedenen Mitglieder zu verteilen.

    \item
    Strafsekunden werden außerdem bei kleineren Vergehen fällig.
    Hierzu zählen folgende Vergehen:
    \begin{enumerate}[a.]
        \item
        Zu frühes Starten oder zu Bpätes beenden des Trinkvorgangs.
        \item
        Minimales Überschäumen des Bieres.
        \item
        Unfaires Verhalten und arglistige Täuschungen.
        Hierunter fallen insbesondere unberechtigtes Trinken, unberechtigtes, böswilliges \glqq{}Stopp\grqq{}-Rufen sowie das Antäuschen von Würfen.
    \end{enumerate}
\end{enumerate}

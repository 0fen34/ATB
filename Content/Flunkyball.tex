\chapter{Flunkyball}
\section{Aussprache}
\begin{enumerate}[label={(\arabic*)}]
    \item
    Flunkyball wird \textipa{["fl\textturnv Nki\textsecstress b\textopeno l]} ausgesprochen.

    \item
    Die Aussprache \textipa{["flUnki\textsecstress bal]} ist möglich, sollte aber vermieden werden.
\end{enumerate}

\section{Vorbereitung}
\begin{enumerate}[label={(\arabic*)}]
    \item
    Die Spielenden teilen sich in zwei etwa gleich große Teams auf.
    Alle Spielenden erhalten eine 0,33l Flasche Bier.

    \item Variationen der Getränke entsprechend~\ref{Allgemeine_Regelungen:Getränke} sind nur nach Absprache mit dem gegnerischen Team möglich.

    Eine Ausnahme bildet hierbei Radler.
    Es ist auch ohne Absprache möglich 0,33l Bier gegen 0,33l Radler zu tauschen.

    \item
    Spielende können mit 0,5l Flaschen spielen. Der dadurch entstehende Nachteil wird nicht ausgeglichen.

    \item
    Die Teams stehen sich in zwei parallelen Reihen im Abstand von zehn bis 15 Schritten gegenüber.
    Entlang der Reihen verteilen sich die Teammitglieder gleichmäßig und stellen ihre Flaschen auf den Boden.
    In der Mitte steht stabil eine zu etwa 25\% mit Wasser gefüllte Plastikflasche (die Zielflasche).

    \item
    Ein Wurfobjekt wird bereitgestellt.
    Je nach Gewicht des Wurfobjekts kann die Wassermenge in der Zielflasche variiert werden um ein optimales Verhältnis zu erhalten.
    Übliche Wurfobjekte umfassen Handbälle, mindestens halb volle Wasserflaschen und Frisbees.
\end{enumerate}

\section{Spielablauf}
\begin{enumerate}[label={(\arabic*)}]    
    \item
    Ein zufälliges Team beginnt mit seinem Wurf.
    Anschließend wirft abwechselnd je ein Teammitglied aus beiden Teams.

    \item
    Wenn nicht anderweitig beschrieben, dürfen die Spielenden die Linie auf denen ihre Biere stehen nicht übertreten und ihre Bierflaschen nicht berühren.

    \item
    Trifft ein Team bei seinem Wurf die Zielflasche und wirft diese um, so beginnt seine Trinkphase.
    Sie endet, wenn das gegnerische Team die Flasche etwa am selben Ort wieder aufgerichtet hat, sich wieder vollständig und mitsamt dem Wurfobjekts hinter der eigenen Reihe der Bierflaschen befindet und deutlich hörbar \glqq{} Stopp\grqq{} gerufen hat.
    Während der Trinkphase dürfen die Mitglieder des entsprechenden Teams ihre Flaschen berühren und auch daraus trinken.

    \item
    Um aus dem Spiel auszuscheiden, halten Spielende ihre Flaschen für einige Sekunden mit der Öffnung nach unten.
    Fließt nichts aus der Flasche, scheiden die Spielenden aus.
    Sind alle Teammitglieder eines Teams erfolgreich ausgeschieden, hat das Team gewonnen.
    Bei einzelnen Tropfen und sehr geringen Mengen Schaum sind die Spielenden des gegnerischen Teams angehalten etwas nachsichtig zu sein.
\end{enumerate}

\section{Das Werfen}
\begin{enumerate}[label={(\arabic*)}]
    \item
    Geworfen wird von hinter der Bierlinie des eigenen Teams von unten.
    Ausnahme bilden Wurfobjekte, die eine abweichende Wurftechnik vorschreiben.

    \item
    Geworfen wird primär auf die Zielflasche.
    Es ist Spielenden allerdings auch erlaubt auf die Bierflaschen des gegnerischen Teams zu zielen.

    \item
    Innerhalb der Teams werfen alle Teammitglieder der Reihe nach, sodass alle Teilnehmenden etwa gleich viel werfen.
    Ausgeschieden Teammitglieder werfen nicht mehr.

    \item
    Gastwürfe von nicht spielenden Personen sind erlaubt, wenn das gegnerische Team sein Einverständnis gibt.
\end{enumerate}

\section{Verteidigen}
\begin{enumerate}[label={(\arabic*)}]
    \item
    Unter Verteidigen versteht sich das Schützen der eigenen Bierflasche oder der Bierflasche eines Teammitglieds davor umgeworfen zu werden.

    \item
    Verteidigt werden darf ab dem Moment, in dem das Wurfobjekt die Hand der Werfer:in verlässt.
    Zu diesem Zweck darf die Bierlinie überschritten werden.
    Mindestens ein Fuß muss aber zu jeder Zeit hinter der Bierlinie auf dem Boden bleiben.

    \item
    Die eigene Bierflasche oder die Bierflasche eines Teammitglieds darf zum Verteidigen nicht berührt werden.

    \item
    Die Bierlinie darf erst zum Zwecke des Aufstellens der Zielflasche und zum Wiedererlangen des Wurfobjekts vollends überschritten werden, sobald die Zielflasche getroffen wurde.
\end{enumerate}

\section{Strafbier}
\begin{enumerate}[label={(\arabic*)}]
    \item\label{Flunkyball:Strafbier:Allg}
    Wird Bier aus der eigenen Flasche verschüttet, so erhält man ein Strafbier.
    Das bisherige Bier ist zügig zu leeren und wird anschließend durch eine neue, volle Flasche ersetzt.

    \item
    Unter~\ref{Flunkyball:Strafbier:Allg} zählt explizit auch Bierschaum.

    \item
    Unter~\ref{Flunkyball:Strafbier:Allg} zählt explizit auch verschüttetes Bier, wenn die Flasche im mit dem Ziel des Ausscheidens umgedreht wird.

    \item
    Werden Regelbrüche festgestellt, wird die Person zunächst verwarnt.
    Bei wiederholtem Regelbruch erhält sie ein Strafbier.
\end{enumerate}

\chapter{Pferderennen}
\section{Vorbereitung}
\begin{enumerate}[label={(\arabic*)}]    
    \item
    Die Asse eines 52er Kartensets werden zur Seite gelegt und die übrigen Karten gut gemischt.
    Sieben zufällige Karten werden verdeckt in einer Reihe auf den Tisch gelegt.
    Die Asse werden offen in einer Spalte orthogonal zu den verdeckten Karten auf den Tisch gelegt, sodass die Karten gemeinsam ein 4 \begin{math}\times\end{math} 7 Feld abstecken.
    Die verdeckten Karten bilden die Seitenlinie des Spielfeldes.
    \begin{verbatim}
    +---+---+----+----+----+----+----+----+
    |   | X |  X |  X |  X |  X |  X |  X |
    | A |   |    |    |    |    |    |    |
    | A |   |    |    |    |    |    |    |
    | A |   |    |    |    |    |    |    |
    | A |   |    |    |    |    |    |    |
    +---+---+----+----+----+----+----+----+
    
    A = Ass
    X = verdekte zufällige Karte
    \end{verbatim}

    \item
    Die Spieler:in setzen auf eines der vier Asse – die \glqq{}Pferde\grqq{}.
    Gesetzt werden Schlucke oder Shots, die direkt selbst getrunken werden.
    Wie viele Schlucke oder Shots gesetzt werden und auf welches Pferd ist den Spielenden überlassen.
\end{enumerate}

\section{Spielablauf}
\begin{enumerate}[label={(\arabic*)}]
    \item
    Nach und nach werden die übrigen Karten aufgedeckt.
    Das entsprechende Pferd mit derselben Farbe darf um einen Platz nach vorne ziehen.
    Im ersten Zug liegt die Karte also neben der ersten verdeckten Karte und nach acht Schritten hinter der letzten.

    \item
    Sind alle Pferde mindestens auf Höhe einer verdeckten Karte, so wird diese aufgedeckt und das entsprechende Pferd wandert einen Schritt zurück.

    \item
    Gewonnen haben die Spielenden, deren Pferd als erstes alle verdeckten Karten hinter sich lässt.

    \item
    Alle Sieger:innen dürfen die doppelte Zahl der gesetzten Schlucke oder Shots unter den Mitspielenden verteilen.
    Alle Verlierer:innen trinken erneut die Zahl ihrer gesetzten Schlucke oder Shots.
\end{enumerate}

\chapter{Pferderennen}
\section{Vorbereitung}
\begin{enumerate}[label={(\arabic*)}]    
    \item
    Die Asse werden aus einem 32er Kartenset zur Seite gelegt und die übrigen Karten gut gemischt.
    Sieben zufällige Karten werden verdeckt in einer Reihe auf den Tisch gelegt.
    Die Asse werden offen ebenfalls in einer Reihe auf den Tisch gelegt, sodass sie orthogonal zu der vorherigen Reihe kurz vor Beginn der vorherigen Reihe liegen.
    Die verdeckten Karten bilden die Seitenlinie des Spielfeldes.

    \item
    Die Spielende setzen auf eines der vier Asse – die \glqq{} Pferde\grqq{}.
    Gesetzt werden Schlücke, die direkt selbst getrunken werden.
    Wie viele Schlücke oder Shots gesetzt werden und auf welches Pferd ist den Spielenden überlassen.
\end{enumerate}

\section{Spielablauf}
\begin{enumerate}[label={(\arabic*)}]
    \item
    Nach und nach werden die übrigen Karten aufgedeckt.
    Das entsprechende Pferd mit demselben Symbol darf um einen Platz nach vorne ziehen.
    Im ersten Zug liegt die Karte also neben der ersten verdeckten Karte und nach acht Schritten hinter der letzten.

    \item
    Sind alle Pferde mindestens auf Höhe einer verdeckten Karte, so wird diese aufgedeckt und das entsprechende Pferd wandert einen Schritt zurück.

    \item
    Gewonnen haben die Spielenden, deren Pferd als erstes alle verdeckten Karten hinter sich lässt.

    \item
    Alle Sieger:innen dürfen die doppelte Zahl der gesetzten Schlücke oder Shots unter den Mitspielenden verteilen.
    Alle Verlierer:innen trinken erneut die Zahl ihrer gesetzten Schlücke oder Shots.
\end{enumerate}

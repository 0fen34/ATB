\chapter{Bier-Pong}
\section{Vorbereitung}
\begin{enumerate}[label={(\arabic*)}]
    \item
    Zwei Teams aus je zwei Spielenden spielen gegeneinander.
    Auf dem Tisch werden pro Team 10 Becher in Pyramidenform aufgestellt.
    Die Basis der Pyramide befindet sich dabei mittig kurz vor den gegenüberliegenden kürzeren Seiten des Tisches und die Becher berühren einander.
    Auf jeder Seite wird \SI{1}{\liter} Bier gleichmäßig auf die Becher verteilt.
    Außerdem wird pro Seite ein mit Wasser gefüllter Becher bereitgestellt, um darin die Bälle zu säubern.
    Gespielt wird mit zwei Tischtennisbällen.

    \item
    Variationen der Teamgrößen sind möglich.
    Die Füllmengen werden entsprechend angepasst.
    Bei ungleichen Teams entscheidet das kleinere Team, ob beide Teams gleich viel Inhalt vorbereiten und wenn ja, an welcher Teamgröße sich orientiert wird.

    \item
    Variationen der Füllmengen sind möglich, wenn sich die Teams darauf einigen.

    \item
    Variationen der Getränke sind entsprechend~\ref{Allgemeine_Regelungen:Getränke} möglich, wenn sich die Teams darauf einigen.
\end{enumerate}

\section{Spielablauf}\label{Bier-Pong:Spielablauf}
\begin{enumerate}[label={(\arabic*)}]
    \item
    Das Auswerfen: Je eine Spieler:in aus jedem Team wirft gleichzeitig mit je einem Ball auf die gegnerische Formation.
    Sie schauen sich dabei in die Augen. Das Auswerfen wird so lange wiederholt bis genau eines der Teams trifft und somit gewinnt.
    In diesem Prozess wechseln sich die Teammitglieder nach jedem Wurf ab.
    Das Sieger-Team beginnt das Hauptspiel und hat am Ende einen Nachwurf.
    Das Verlierer-Team hat keinen Nachwurf.

    \item
    Das Hauptspiel: Die Teams werfen in einem Zug nacheinander je zwei Bälle auf die gegnerische Formation am gegenüberliegenden Tischende.
    Um als Treffer zu zählen, muss der Ball im Becher zur Ruhe kommen.
    Die Becher bleiben mitsamt sich womöglich darin befindenden Bällen bis zum Ende eines Zuges unberührt stehen.
    Der Zug endet, wenn das werfende Team keine Würfe mehr offen hat.

    \item
    Das Spielende: Wurden alle Becher eines Teams getroffen und hat dieses Team kein Wurfrecht mehr, so hat es das Spiel verloren.
    Das Verlierer-Team teilt die übrigen Becher des Siegerteams unter sich auf und trinkt diese zeitnah.

    \item
    \glqq{}Sudden Death\grqq{}: Wurden alle Becher aller Teams getroffen ohne ein klares Siegerteam, wird der Sieg durch Sudden Death entschieden.
    Auf jeder Seite wird ein einzelner Becher aufgestellt und zur Hälfte gefüllt.
    Die Teams werfen abwechselnd mit je einem Ball.
    Wer als erstes trifft ohne, dass das andere Team ebenfalls im Nachwurf trifft, gewinnt das Spiel.
\end{enumerate}
    
\section{Das Werfen}
\begin{enumerate}[label={(\arabic*)}]
    \item\label{Bier-Pong:Werfen:Ellenbogen}
    Beim Werfen muss der Ellenbogen des Wurfarms ständig hinter der Tischkante bleiben.

    \item
    Wird~\ref{Bier-Pong:Werfen:Ellenbogen} verletzt, so wird zunächst eine Verwarnung ausgesprochen und der Wurf wiederholt.
    Bei wiederholter Verletzung muss von der Werfer:in ein beliebiger eigener Becher getrunken werden.

    \item
~\glqq{}Trickshots\grqq{} müssen angekündigt werden, um als solche zu zählen.
    Sie müssen sich signifikant von normalen Würfen unterscheiden.
    Trifft ein Trickshot, so muss ein zusätzlicher Becher getrunken werden.

    \item
~\glqq{}Bouncer\grqq{} müssen nicht angekündigt werden.
    Prallt ein Ball von der Tischplatte ab, bevor er einen Becher trifft, so müssen ebenso viele zusätzliche Becher getrunken werden, wie der Ball aufgeprallt ist.
    Bouncer dürfen entsprechend~\ref{Bier-Pong:Verteidigen} abgewehrt werden.

    \item
    Als \glqq{}Airball\grqq{} zählt jeder Wurf außer Trickshots, der am Tisch vorbei geworfen wird, ohne ihn oder etwas darauf zu berühren.
    Wird der Airball vom verteidigenden Team vor deren nächsten Wurf erkannt, hat die werfende Person die Wahl, entweder einen Shot zu trinken oder bis Spielende ein Kleidungsstück auszuziehen.
    Socken oder Ähnliches zählen gemeinsam als ein Kleidungsstück.
    Schmuck zählt nicht zur Kleidung.
    Die Wahl steht der werfenden Person frei.

    \item\label{Bier-Pong:Werfen:Ankündigen}
    Es ist möglich, einen konkreten Becher als Ziel anzugeben.
    Wird der Becher getroffen, muss ein zusätzlicher Becher getrunken werden.
    Wird dieser Becher nicht getroffen, trinkt stattdessen die werfende Person einen eigenen Becher ihrer Wahl.

    \item
    Ziele gemäß~\ref{Bier-Pong:Werfen:Ankündigen} können nur angekündigt werden, wenn noch mindestens vier Becher zu treffen sind.

    \item
    Gastwürfe von nicht spielenden Personen sind erlaubt, wenn das gegnerische Team sein Einverständnis gibt.
    Diese Person zählt für die Dauer des Zuges als Teammitglied des werfenden Teams.

    \item
    Sammeln sich im Laufe eines Zuges zusätzliche zu trinkende Becher an, so werden die zu trinkenden Becher am Ende des Zuges vom werfenden Team bestimmt und zählen ab diesem Moment als getroffen.
\end{enumerate}

\section{Verteidigen}\label{Bier-Pong:Verteidigen}
\begin{enumerate}[label={(\arabic*)}]
    \item
    Wird unerlaubt verteidigt, so wird der Wurf wiederholt und es muss ein zusätzlicher Becher getrunken werden.

    \item
    Verschüttete, eigene Becher müssen wieder aufgefüllt und anschließend getrunken werden.
    Werden beim Verteidigen gegnerische Becher verschüttet, so werden diese wieder aufgefüllt und es muss die entsprechende Anzahl eigener Becher getrunken werden.
    Die Auswahl erfolgt hierbei frei.

    \item
    Als Verteidigen gilt jedes signifikante Einwirken auf Ball, Tisch oder ungetroffene Becher während eines Zuges.

    \item
    Verteidigen ist grundsätzlich nicht erlaubt.
    Fingern und Blasen sind ausdrücklich nicht erlaubt.

    \item
    Bälle, die bereits Kontakt mit einem anderen Objekt hatten dürfen verteidigt werden.
    Hierzu zählen insbesondere Bouncer.
    Trickshots dürfen nicht verteidigt werden.

    \item
    Airballs dürfen gefangen werden, sobald sie über die Tischkante geflogen sind.
    Einen Airball vorher abzufangen zählt als unerlaubte Verteidigung.
\end{enumerate}

\section{Getroffene Becher}
\begin{enumerate}[label={(\arabic*)}]
    \item
    Getroffene Becher werden direkt nach Zugende aus der Formation entfernt und zeitnah geleert.
    Üblicherweise trinken die Teammitglieder abwechselnd.
    Sie können sich aber auch auf eine alternative Regelung einigen.

    \item
    Durch den Wurf umgeworfene Becher zählen nicht als getroffen.
    Sie werden nach Ende des Zuges wieder neu befüllt und an ihrer vorherigen Position aufgestellt.

    \item
    Trifft ein Team einen noch nicht geleerten Becher, der in einem vorherigen Zug getroffen wurde, so hat es das Spiel sofort gewonnen.
    Dies stellt ein alternatives Spielende zu dem in~\ref{Bier-Pong:Spielablauf} geschilderten dar.
    Es gibt keinen Nachwurf.
\end{enumerate}

\section{Zusammenstellen}
\begin{enumerate}[label={(\arabic*)}]
    \item
    Hat ein Team zu Beginn eines Zuges noch einen, drei oder sechs Becher zu treffen, so darf es die Becher zusammenstellen lassen.
    Die Becher werden dabei wieder zu einer Pyramiden-Formation aufgestellt.
    Um herunterfallende Becher zu vermeiden, ist wieder ein kleiner Abstand zur Tischkante einzuhalten.

    \item
    Fordert ein Team zu einem anderen Zeitpunkt als zu Beginn des eigenen Zuges das Zusammenstellen, so muss das gegnerische Team dem Wunsch weder direkt, noch unaufgefordert zum angemessenen Zeitpunkt nachkommen.

    \item
    Verrutschte Becher dürfen zwischen Zügen wieder gerade gerückt werden.

    \item
    Sonstiges Bewegen von nicht getroffenen Bechern zwischen Zügen ist nicht erlaubt.
\end{enumerate}

\section{Besondere Situationen}
\begin{enumerate}[label={(\arabic*)}]
    \item
    \glqq{}Balls Back\grqq{}: Trifft ein Team in einem Zug mit beiden Bällen, so kommt es direkt noch einmal zum Zug.
    In diesem Fall hat es aber keine Möglichkeit die Becher zusammenzustellen.

    \item
    Trifft ein Team in einem Zug mit beiden Bällen in denselben Becher, müssen drei zusätzliche Becher getrunken werden.
    Die Becher werden vom werfenden Team frei gewählt.

    \item
    \glqq{}1UP\grqq{}: Rollt ein geworfener Ball auf dem Tisch zurück, so darf das werfende Team ihn wieder an sich nehmen, sobald er die Mittellinie überschritten hat.
    Der Ball darf in Form eines Trickshots erneut geworfen werden.
    Im Falle eines Treffers muss aber kein zusätzlicher Becher getrunken werden.

    \item
    Wird auf einem Tisch mit Löchern für die Becher gespielt, so kann es dazu kommen, dass der Ball ohne eine einzige Berührung durch eines der Löcher geworfen wird.
    Dieser Wurf zählt nicht als normaler Airball, sondern, als hätte jedes Teammitglied des verteidigenden Teams einen Airball geworfen.
\end{enumerate}

\chapter{Busfahren (Gruppe) / Pyramide}
\section{Vorbereitung}
\begin{enumerate}[label={(\arabic*)}]
    \item Auf dem Tisch werden 15 Spielkarten aus einem 32er Kartenset in Pyramidenform verdeckt ausgelegt.
    \item Die Spielenden erhalten je drei Spielkarten verdeckt. Gespielt wird mit linearen Bestrafungen (siehe~\ref{Busfahren_group:Spielablauf:Bestrafung}).
    \item Bei mehr als fünf Spielenden werden nur 10 Spielkarten ausgelegt. Gespielt wird mit exponentiellen Bestrafungen (siehe~\ref{Busfahren_group:Spielablauf:Bestrafung}).
    \item Die übrigen Karten werden verdeckt bei Seite gelegt.
\end{enumerate}

\section{Spielablauf}\label{Busfahren_group:Spielablauf}
\begin{enumerate}[label={(\arabic*)}]
    \item
    Alle Spielenden machen sich mit ihren Karten vertraut.
    Hierzu dürfen sie sich alle eigenen Karten anschauen und nach Belieben sortieren.
    Anschließend legen sie die Karten verdeckt nebeneinander vor sich auf den Tisch.

    \item
    Die Karten der Pyramide werden nun nach und nach aufgedeckt.
    Zunächst werden die Karten der längsten Seite aufgedeckt und von dort aus der Reihe nach bis zur einzelnen Karte an der Spitze.
    Nach jeder aufgedeckten Karte wird kurz pausiert.
    Spielende, die Karten mit demselben Symbol vor sich ausliegen haben, dürfen diese nun aufdecken und entsprechend~\ref{Busfahren_group:Spielablauf:Bestrafung} Schlucke unter allen Spielenden verteilen.

    \item Schlucke werden in der Reihenfolge verteilt, wie die Spielenden angekündigt haben verteilen zu dürfen.

    \item\label{Busfahren_group:Spielablauf:Bestrafung}
    Aus der Reihe, in der die aufgedeckte Karte in der Pyramide lag, ergibt sich wie viele Schlucke verteilt werden dürfen
    \begin{enumerate}[a.]
        \item
        Im Spielmodus linearer Bestrafung darf in der ersten Reihe ein Schluck verteilt werden.
        Mit jeder zusätzlichen Reihe darf ein Schuck mehr verteilt werden.
        \item
        Im Spielmodus exponentieller Bestrafung darf in der ersten Reihe ein Schluck verteilt werden.
        Mit jeder zusätzlichen Reihe dürfen doppelt so viele Schlucke verteilt werden.
    \end{enumerate}

    \item 
    Hat jemand mehrfach dieselbe Karte vor sich ausliegen, so darf die Person für jede Karte einzeln Schlucke verteilen.

    \item
    Karten, für die Schlucke verteilt wurden, werden abgelegt.

    \item\label{Busfahren_group:Spielablauf:Ziehen}
    Deckt jemand eine falsche Karte auf, so trinkt die Person selbst so viele Schlucke wie sie hätte verteilen dürfen.
    Die Karte ist anschließend aus dem Spiel und die Person zieht eine neue zufällige Karte aus den nicht verteilten Karten.
    Sind diese aufgebraucht wird zufällig aus den abgelegten Karten gezogen.

    \item
    Die Person, die am Ende die meisten Karten auf der Hand hat, fährt Bus entsprechend~\ref{Busfahren_solo}.
    Haben mehrere Personen gleich viele Karten auf der Hand wird durch Sudden Death entschieden, wer Bus fährt.
    Im Sudden Death sagen die Spielenden abwechselnd die Farbe von Karten vorher.
    Die erste Person, die einen Fehler macht, während die andere Person richtig liegt, fährt Bus.
    Die jüngere Person beginnt mit dem Raten.
\end{enumerate}

\section{Flunkern}
\begin{enumerate}[label={(\arabic*)}]
\item
Vor Spielbeginn können die Spielenden sich darauf einigen, dass mit Flunkern gespielt wird.

\item
In diesem Spielmodus werden Karten nicht direkt aufgedeckt, wenn man behauptet, mit ihnen Schlucke verteilen zu dürfen.
Wer verpflichtet wird zu trinken, darf die Behauptung anzweifeln.
Die entsprechenden Karten werden dann aufgedeckt und entsprechend~\ref{Busfahren_group:Spielablauf:Ziehen} durch neue Karten ersetzt.

\item
Wird eine Karte zurecht angezweifelt, muss die Lügner:in die entsprechenden Schlucke selbst trinken.
Wird eine Karte fälschlicherweise angezweifelt, muss die anzweifelnde Person die doppelte Zahl an Schlucken trinken.

\item
Im Spielmodus mit Flunkern wird nicht entsprechend~\ref{Busfahren_solo} Bus gefahren.
\end{enumerate}

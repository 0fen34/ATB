\chapter{Busfahren (Einzel)}\label{Busfahren_solo}
\section{Vorbereitung}
\begin{enumerate}[label={(\arabic*)}]
    \item
    Gespielt wird von einer einzelnen Spieler:in.
    Im Spieljargon \glqq{}fährt die Person Bus\grqq{}.

    \item
    Unterstützt wird sie von der Dealer:in.
    Die Dealer:in hält ein durchgemischtes 32er Kartenset mit den Kartenrücken nach oben vor sich.
    Die Dealer:in wird \glqq{}Busfahrer\grqq{} genannt.
\end{enumerate}

\section{Spielablauf}
\begin{enumerate}[label={(\arabic*)}]
    \item
    Die Aufgabe der Spieler:in ist es, die als nächstes gelegten Karten vorherzusagen.
    Errät sie vier Karten in Folge richtig, gewinnt sie und das Spiel endet.
    
    \item
    Für jede falsch vorhergesagten Karte trinkt die Spieler:in einen Schluck.
    
    \item
    Die Karten müssen in entsprechend der folgenden Sequenz vorhergesagt werden.
    Bei einer falsch vorhergesagten Karte beginnt die Sequenz von Neuem:

    \begin{enumerate}[a.]
        \item\label{Busfahren_solo:Spielablauf:Fragen:RS}
~\glqq{}Rot oder Schwarz?\grqq{}:
        Ist die Karte rot (Karo, Herz) oder schwarz (Kreuz, Pik)?
        \item
~\glqq{}Höher oder tiefer?\grqq{}:
        Ist die Karte höher oder tiefer als die vorherige Karte?
        Es kann auch geraten werden, dass die Karte gleich ist.
        \item
~\glqq{}Dazwischen oder draußen?\grqq{}:
        Liegt die Karte zwischen den beiden vorherigen Karten oder außerhalb.
        Es kann auch geraten werden, dass sie auf einer der Grenzen liegt.
        \item
~\glqq{}Rot oder Schwarz?\grqq{}:
        Siehe oben.
    \end{enumerate}

    \item
    Die Karten zählen von der niedrigsten zur höchsten folgendermaßen: 7, 8, 9, 10, B, D, K, A.
\end{enumerate}

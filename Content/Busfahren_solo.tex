\section{Busfahren (Einzel)} \label{Busfahren_solo}
\subsection{Vorbereitung}
\paragraph{}
Gespielt wird von einer einzelnen Spieler:in.
Im Spieljargon \glqq fährt die Person Bus\grqq{}.

\paragraph{}
Unterstützt wird sie von der Dealer:in.
Die Dealer:in hält ein durchgemischtes 32er Kartenset mit den Kartenrücken nach oben vor sich.
Die Dealer:in wird \glqq Busfahrer\grqq{} genannt.


\subsection{Spielablauf}
\paragraph{}
Die Aufgabe der Spieler:in ist es, die als nächstes gelegten Karten vorherzusagen.
Errät sie vier Karten in Folge richtig, gewinnt sie und das Spiel endet.

\paragraph{}
Für jede falsch vorhergesagten Karte trinkt die Spieler:in einen Schluck.

\paragraph{}
Die Karten müssen in entsprechend der folgenden Sequenz vorhergesagt werden.
Bei einer falsch vorhergesagten Karte beginnt die Sequenz von Neuem:
\subparagraph{} \label{Busfahren_solo:Spielablauf:Fragen:RS}
Rot oder Schwarz? – Ist die Karte rot (Karo, Herz) oder schwarz (Kreuz, Pik)?
\subparagraph{}
Höher oder tiefer? – Ist die Karte höher oder tiefer als die vorherige Karte?
Es kann auch geraten werden, dass die Karte gleich ist.
\subparagraph{}
Dazwischen oder draußen? – Liegt die Karte zwischen den beiden vorherigen Karten oder außerhalb.
Es kann auch geraten werden, dass sie auf einer der Grenzen liegt.
\subparagraph{}
Rot oder Schwarz? – Siehe \ref{Busfahren_solo:Spielablauf:Fragen:RS}.

\paragraph{}
Die Karten zählen von der niedrigsten zur höchsten folgendermaßen: 7, 8, 9, 10, B, D, K, A